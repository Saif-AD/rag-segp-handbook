\chapter{Small group project}
\label{ch:small-group-project}

\section{Group allocation}
\subsection{Approach}
\tfaq[1]{0mm}{How are students allocated to teams in the small group project?}{Small~group~project}
\index{group~allocation!small~group~project}
\index{small~group~project!group~allocation}
Prior to the start of the small group project, you will be allocated to a team of four or five people.  This allocation is not random.  However, we have no reliable data to predict how well people will work in a team.  Academic attainment tends to be a poor predictor.   

Instead of trying to predict how people will work in a team, we ask that students make a commitment to their future team, expressing how long and how intensely they are seek to collaborate with team members.  Students are then allocated to a team of people who have made a roughly similar commitment.  The expectations arising from the commitment you have made will be enforced through the individual marking scheme.  Specifically, major downwards corrections will be made where team members fail to meet up to their commitments (along the lines set out in the marking scheme).

To be allocated to a team, you must complete a registration form (available via KEATS) by the registration deadline.  You will only be allocated to a team if you complete this form.  The form will ask a number of question and remind you of the commitments you are making.  However, the key pieces of information are as follows:
\begin{itemize}
\item The project schedule you wish to adopt.  You will have a chance to select to work along a six week (normal/full) schedule, a four week (shortened) schedule, or three/two week (extra short) schedule.  All schedules work towards the same deadline, but the longer schedules start earlier.  The six week schedule requires that you organise during reading week and the project proper immediately after reading week.  We aim to allocate you to a team where everyone chose the same schedule.  You must achieve the pre-requites for participating in the small group project, as set out in Section~\ref{sect:sgt:prerequisites} \emph{before} the scheduled start date of the project\footnote{If you struggle to keep up with training in Semester 1, give yourself enough time to catch up by choosing a shorter schedule.}.
\item The number of hours you dedicate to working on the small group project per week.   You have a choice of 15 hours (intense), 10 hours (normal), and 7 hours (relaxed).  You must dedicate this number of hours in each calendar week (Monday--Sunday) of your chosen project schedule.  This time includes team meetings and other work related to the small group project, but excludes independent learning and training.  Beware that a higher intensity schedule will leave less time for other modules.  It may not be possible to ensure that everyone in your team has selected the same option here, but we aim to achieve as close a match as possible.
\item \emph{Optionally}, the gender, and ethnicity you identify with\footnote{We will not ask about disability.  If you have a \ac{KIP} and it is shared with module organisers, you will be contacted separately to address any requirements set out in there.}. When provided, we aim to avoid allocating you to a team where you stand out due to your gender identity or ethnicity.  Where possible, this is achieved by ensuring you are not the sole member with a given gender identity or ethnicity in your team.   Otherwise, we aim to introduce significant diversity in your team.  Please note that your gender and ethnicity can only be taken into account if your provide that information.
\item Confirmation that you have read and understood the expectations.
\end{itemize}

There will be two rounds of allocation.  In the first round, everyone who submitted the registration form on time will be allocated to a team.  A second round with a later registration deadline is organised for everyone who failed to submit their registration form on time.\index{group~allocation!missed~registration~deadline}.  The six week schedule is only available as an option for on-time registration.  If you do not submit a registration form before the second deadline, you cannot be allocated to a team.


\subsection{Rationale}
\tfaq[1]{0mm}{Why do you allocate students in the small group project based on student commitment, gender, and ethnicity?}{Small~group~project}
Forming groups for student group projects is a challenging task.  The current system has emerged from many years of experience and experimentation aimed at finding a better way to achieve this.

Over the years, we have made many attempts to group students based on academic attainment.  These have included using Year 1 marks, in-term tests, and self-assessment.  All these approaches have been severely flawed for two reasons.  Firstly, while poor academic attainment tends to be a good predictor that  student will struggle in a team, high academic attainment is a poor predictor for how well a student will fit in a team.  As high academic attainment is much more common in Year 2 than poor academic attainment, academic attainment is a poor predictor for the team a student should be allocated to.  Secondly, the most common challenge students face in group project is a mismatch in expected engagement.  Specifically, if there are significant difference in expectations regarding the number of weeks students want to commit to the project, or the number of hours per week team members wish work, then some members of that team will struggle.  The allocation scheme aims to address that.

Of course, that may raise questions about ability.  However, the small group project is not intended to be overly technically challenging\footnote{You have Practical Experiences of Programming (5CCS2PEP) to do that!}.  In fact, every student in the class is expected to work to the same standard.   Following sufficient training, designing, building, and testing CRUD operations using Django and related Python packages, and using standard software engineering tools should not especially technically demanding for student of the caliber that reach Year 2 of Computer Science course at King's College London.  Developing this type of software to high quality standards will probably come more natural to some student than to others, but software quality is collective team responsibility, achieved though careful organisation of the work, and good collaboration and communication.

\section{Prerequisites to participate}
\label{sect:sgt:prerequisites}
\tfaq[1]{0mm}{What knowledge and skills must team members have acquired before starting the small group project?}{Small~group~project}
In the small group project, you will need to develop a Django application in collaboration with your peers.  Before you are able to participate in such a project, it is essential that you possess the basic skills needed for collaborative software development with the technology stack we are using.  In particular, \emph{before you start working on the project}, you must be able to:
\begin{itemize}
\item Apply fundamental computer literacy skills to set up and manage your development machine, and deploy a Django application using a service such as PythonAnywhere.
\item Use the range of development tools we use in this module for Django development.  In particular, you need to be able to use git/GitHub version control software to share your work with the team in a professional manner, use version control and Team Feedback tools to ensure that your work is correctly attributed to you, integrate work from branches with that of the main, and use continuous integration to ensure that problems arising from integration are spotted early.  In other words, you need to be able to use the tools the project requires.
\item Build and refine models, views, templates, and related components required for a full-stack implementation of \ac{CRUD} operations, as well as comprehensive automated test suites of what you have implemented.  This should enable about to produce realistic estimates of the effort involved in a task, and take on small, vertical and complete these within a week.  You also require sufficient knowledge of Django in order to participate meaningfully in code inspection/review exercises.  In other words, you team mates have a reasonable expectation that you can collaborate in all aspects of the development work and produce usable work.
\item Manage your time effectively, so that \emph{in every week of the team's development period, you can dedicate about a quarter of your time to the small group project}.  Every piece of work you produce needs to be reviewed and integrated by your team mates.  As the team learns the cadence at which it can produce work, new tasks need to be assigned.  To enable the team to do this, you need to be produce work for the team consistently in every week you participate in the project.  Beware that mark scheme is designed to enforce this expectation.
\item Recognise how the project management techniques you have learned the first half of Semester 1 apply to the project, and apply key skills including effective communication, agile planning/scheduling, effort estimation, and risk management.
\end{itemize}
It takes a considerable amount of time to acquire these skills.  Normally, you need to start the learning process from the very start of Semester 1 in order to be ready to start the small group project immediately after reading week.  You also need to organise your work on other modules in such a way that you can engage with the small group project team on a weekly basis.  Of course, we all learn at a different pace.  Therefore, you can participate in the small group project on different schedules.

\section{Project schedules}
\tfaq[1]{0mm}{Why are there small group project?}{Small~group~project}
Software development projects are exposed to risks and challenges.  To manage those risks and tackle those challenges, teams need time.  With more time, teams have more space to recognise risks and challenges, discuss them, formulate a plan, and execute that plan.  The small group project is designed to require six weeks of low intensity work.  Teams are strongly recommended to use the full six weeks available to them, though working at a slow pace (up to 10 hours per week, or 25\% of the working week), leaving time to complete other work concurrently.

In practice, many students do not follow that recommendation for a variety of reasons.  Some are simply not ready to participate in the small group project when the assignment is released because they lack the prerequisite skills and knowledge.  Although I would recommend against this, some students choose to focus all of their effort on a single coursework at any one time.  

Members of the same team choosing to work to different schedules is a major cause of friction in teams.  To address this, three different teamwork schedules are proposed.  As part of the allocation process, you must choose one of the schedules and you will be allocated to a team of students who have selected the same schedule.  Therefore, by choosing a schedule, you are making a commitment to the team that you join.  Each schedule comes with a set of expectations that are enforced through the marking scheme.  Think carefully about the implications and risks of selecting a particular schedule.  Once you are allocated to a team, it will no longer be possible to make changes.  If you find yourself unable to meet the expectations of a schedule after allocation, the only remaining option will be to apply to defer the small group project by means of a \ac{MCF}.

\subsection{Six week schedule}
\tfaq[1]{0mm}{What are the pros and cons of choosing the six week schedule?}{Small~group~project}
As the name suggests, the six week schedule spreads out the build of the software over a period of six weeks from the first week of teaching after Reading Week until the deadline.  You are strongly recommended to choose this schedule, \emph{provided you are capable of meeting the expectations your teammates will have of you}.  The six week schedule has important advantages:
\begin{itemize}
\item Because the project duration is spread out, the development pace can be slowed down.  This gives teams ample time to design the product, control software quality (through code inspections and review), and correct problems as they arise.
\item When a new team starts collaborating, planning is extremely hard because it is nearly impossible to predict what team members are capable of producing in a given amount of time.  Over time, you observe the work your team members produce and you have opportunities to inspect the code quality.  This makes it easier to plan work.  The six week schedule, you have the time to learn about the capabilities of your team mates and improve the quality of your plans.
\item Projects are inherently risky.  Under a six week schedule, the team has time to recognise any risks that materialise and respond.
\item If some of your team's initial arrangements do not work for you, the team has time to incorporate a mid-project review and change its ways of working.
\end{itemize}

To join a team following the six week schedule, you must meet the following requirements:
\begin{itemize}
\item You must meet all the prerequisites specified in Section \ref{sect:sgt:prerequisites} by the end of reading week.  It is essential that you can independently implement useful and fully tested user stories in Django, contribute them via a git branch, ensure that authorship is attributed correctly, inspect code, and integrate a git branch in to the main after receiving permission from your team to do so.  You must be capable of doing this from the very start, without requiring help from team mates or time to figure this out.
\item You must be able to attend a kick off meeting during Reading Week so that the team can hit the ground running when the small group project starts immediately after Reading Week.
\item You must be able to balance your work on the small group project with other coursework and study commitments, including the substantial Practical Experiences of Programming (5CCS2PEP) coursework taking place concurrently.
\item During every one of the six weeks of the project, every team member will be expected to attend the team meetings, take responsibility for an incomplete task, produce a solution by the next week (at the latest!), and present their result and solutions by the next week.  This must manifest itself in the form of meaningful weekly code contributions attributed to you on Team Feedback, that eventually make their way to the main branch.  Normally, failing to meet these expectations in one of the six weeks will be permitted without MCF.
\end{itemize}
It is your individual responsibility to meet these expectations.  Individuals choosing to join teams operating the six week schedule, but failing to meet these expectations can be very disruptive.  To disincentivise this, the individual marking scheme will be applied strictly.  For example, failing to produce meaningful weekly code contributions in all (but one) of the project weeks leads to major mark corrections.  If, for instance, all team members contribute in only four weeks of the project, then all team members will see their mark reduced because none met the requirements for the six week schedule!

Teams operating a six week schedule must ensure the following:
\begin{itemize}
\item The team must not delay the start of the project under any circumstances!  In particular, lack of engagement by one or two team members is not a reason to postpone the project.  Everyone in your team will have signed up to a six week schedule, so the team should be enforcing that.  If certain members do engage belatedly, manage their late integration into the team carefully.  Only assign tasks at an in-person meeting.  Do not rely on them to complete important task (especially not testing other people's work).  Assign small tasks to be completed very quickly.  Monitor progress.  From the outset, agree what will happen if work is not delivered.
\item The project should be treated as a marathon, not a sprint.  The team needs to pace itself.  Rushing large coding tasks is not going to be sustainable week after week, and this would defeat the purpose of spreading out work over six weeks.  You will be doing other work alongside this project, so constant communication within the team is also unsustainable.  
\item Take the time to agree sensible working practices at the outset.  You will be doing other work alongside this project.  Therefore, constant engagement with the project is excessive.  Set out a regular schedule of meetings.  Agree normal work hours, and on what days and times team members should be able to switch off from the project.  It is also good practice to reserve some time to review these practices at the midpoint of the project.  Some arrangements may not work out as intended, and the team should have an opportunity to review them.
\end{itemize}

Beware that it is only sensible to participate in the small group project on a six week schedule if you are certain you are capable of meeting the schedule's requirement by the end of Reading Week.  Your mark will be adversely affected if this is not the case.  If you select this schedule in the group allocation form, you will be allocated to a six week schedule team and it will not be possible to change teams after allocation.

\subsection{Four week schedule}
\tfaq[1]{0mm}{What are the pros and cons of choosing the four week schedule?}{Small~group~project}
The four week schedule is a shortened schedule whereby the project start is delayed about two weeks.  The team should organise a kick-off meeting and an initial planning meeting some time in the two teaching weeks following Reading Week, with a view to start coding immediately after this two week period has ended.

To join a team following the four week schedule, you must meet the following requirements:
\begin{itemize}
\item You must meet all the prerequisites specified in Section \ref{sect:sgt:prerequisites} within two weeks after reading week.  It is essential that you can independently implement useful and fully tested user stories in Django, contribute them via a git branch, ensure that authorship is attributed correctly, inspect code, and integrate a git branch in to the main after receiving permission from your team to do so.  You must be capable of doing this from the very start, without requiring help from team mates or time to figure this out.
\item You must be able to attend a kick off meeting and a planning meeting with your team mates during the two teaching weeks following Reading Week.  It is important that you participate in these meetings so that the team can incorporate you in their plans.
\item The four week schedule gives you a little bit more time to prepare to contribute to the small group project.  Use this time well to ensure you catch up with the Software Engineering Group Project materials, and complete as much as possible of other significant coursework and study commitments.  Pay particular attention to the  substantial Practical Experiences of Programming (5CCS2PEP) coursework.
\item During every one of the four weeks of the project, every team member will be expected to attend the team meetings, take responsibility for an incomplete task, produce a solution by the next week (at the latest!), and present their result and solutions by the next week.  This must manifest itself in the form of meaningful weekly code contributions attributed to you on Team Feedback, that eventually make their way to the main branch.  Normally, failing to meet these expectations in one of the four weeks will be permitted without MCF.  However, you will need to catch up.
\item You must ensure that the data collected by Team Feedback about your participation and code contributions is correct by the time of submission.  Consult the Team Feedback Help pages for advice is this is not the case.
\end{itemize}

Teams operating a four week schedule must ensure the following:
\begin{itemize}
\item The team must start the project on time: not too soon and not too late.  Specifically, the kick-off and planning meetings need to take place in the two weeks following reading week.  The team must be coding throughout the final four weeks of Semester 1.  
\begin{itemize}
\item Do not delay the start of the project under any circumstances!  In particular, lack of engagement by one or two team members is not a reason to postpone the project.  Everyone in your team will have signed up to a four week schedule, so the team should be enforcing that.  If certain members do engage belatedly, manage their late integration into the team carefully.  Only assign tasks at an in-person meeting.  Do not rely on them to complete important task (especially not testing other people's work).  Assign small tasks to be completed very quickly.  Monitor progress.  From the outset, agree what will happen if work is not delivered.
\item Do not start coding prematurely.  This unfair to others who have signed up expecting a four week schedule, not a five or six week one.  Premature code contribution will be ignored in assessing individual code contributions.
\end{itemize}
\item The project should be treated as a marathon, not a sprint.  The team may be tempted to compensate for the shortened development period by attempting to do more work in each of the four weeks.   While the team can certainly take that approach try and produce six weeks worth of functionality, there is a significant risk that this will undermine code quality and testing.  In general, it will be better to scale down the ambition a little, even if the team is prepared to work at a faster pace than a team on a six week schedule.
\item Make sure to develop tests and inspect code throughout each of the four weeks of the project.  Resist the temptation to get ahead after a later start by focussing solely on functionality.  Such a practice would undermine testing and code quality, limiting your mark irrespective of how much functionality the team managed to implement.
\item Take the time to agree sensible working practices at the outset.  You will be doing other work alongside this project.  Therefore, constant engagement with the project is excessive.  Set out a regular schedule of meetings.  Agree normal work hours, and on what days and times team members should be able to switch off from the project.
\end{itemize}

The four week schedule is a compromise approach between a six week and a three/two week schedule.  It provides more time to prepare for participation in the group project, whilst also leaving time for quality control, testing, building at least some experience of what team members can produce in a week.  Note that to participate on a four week schedule, you will have to be able to contribute within two weeks after reading week. You will also have to plan ahead as there will still be some overlap with the 5CCS2PEP coursework.  Your mark will be adversely affected if this is not the case.  If you select this schedule in the group allocation form, you will be allocated to a four week schedule team and it will not be possible to change teams after allocation.

\subsection{Three/Two week schedule}
\tfaq[1]{0mm}{What are the pros and cons of choosing the three/two week schedule?}{Small~group~project}
The three/two week schedule condenses the small group project into the shortest possible time period.  The idea of this schedule is that students complete all other coursework due during Semester 1 before starting the group project.  They then focus all their attention on the small group project.  \emph{The three/two week schedule is \textbf{not} recommended because it is inherently risky}.  It is offered because, every year, there are teams that adopt this schedule, irrespective of recommended practice.  If you choose to work in this way, familiarise yourself with the risks.  Note that if you chose to expose yourself to these risk, it is your responsibility to mitigate them and live with their impact!

The risks of this shortened project schedule include the following:
\begin{itemize}
\item Relatively minor setbacks can have a disproportionate impact on your participation in the project.  For example, a bout of influenza or a bad cold can prevent you from participating in the project long enough that you will need to defer.  As you would be completing the project in late Autumn, this is a very real possibility.
\item Quality control ensuring high standards of design, code cleanliness, and testing takes time.  Normally, code is produced along with an automated test suite and then refined and improved.  By shortening the project duration to only two to three weeks, the team leaves itself little time to review work in order to ensure standards are upheld.  Moreover, by completing the project closer to the deadline, the team may feel it needs to rush work.  Consequently, quality standards are likely suffer.
\item Planning and scheduling will be difficult.  Because the project duration is so short, teams do not have time to experience the standard and timeliness of work each team member is capable of producing, and then use that experience to plan work in the next cycle.  Once tasks are allocated, it will be difficult to predict whether and to what extent these will be completed.  Some anticipated features may not be deliverable.
\item In a longer project, trust is developed in a team by team members delivering good work in a timely manner.  With a severely shortened project, there is little time for this.  It will be more difficult to build trust in your team.
\end{itemize}
To mitigate these risks, teams can try to adopt the following practices.
\begin{itemize}
\item Organise a kick-off meeting as soon as possible after team allocations are released.  Even though you will not start working on the project any time soon, it will be difficult to convene the team if you do not do this.  At this meeting, you will agree when you will start the project and schedule and date, time, and place for your team's project initiation meeting.  That will be the start of the project.  Chances are that multiple team members do not engage with requests to organise a kick-off meeting.  If so, proceed nonetheless, record your meeting minutes (and their absence) on Team Feedback, and record what you have agreed.
\item It is a good idea for someone to send a reminder of the project initiation meeting about a week before it is due to take place. Irrespective of whether a reminder is sent, everyone should be attending the project initiation meeting.
\item From the outset and at all times during the project, limit your ambitions related to the scope of the project.  Attempting to implement too much functionality is going to affect the quality of your project and this, in turn, may cap your eventual mark (irrespective of how much functionality you produced).
\item Start quality control from the outset.  Testing, and code inspections/reviews cannot be delayed.
\end{itemize}

\subsection{General considerations}
\paragraph{Assessment}
\tfaq[1]{0mm}{Will choosing a particular schedule affect my mark or how I will be assessed?}{Small~group~project}
The schedule you choose only affects the development period of the project.  The length of a team's development period is a factor in assessing major mark corrections for a team (see Section~\ref{sect:sgp:major-mark-correction}).   Otherwise, all teams are marked in exactly the same way.

A team that develops their small group project on a longer schedule will have more time to produce work, quality control the software being produced, and respond to emerging risks.  Therefore, it stands to reason that, on balance, teams adopting a longer schedule will tend attain higher team mark.\IncludeBeforeDate{2026-07-01}{\footnote{We have never employed schedule-based allocation before, so this is purely speculative at this point.}}~  However, the expectations imposed on individual students in teams following a shorter schedule tend to be considerably less demanding.  If you cannot meet the expectations of a particular schedule, your individual mark will be affected in three ways.  Firstly, you will lose one or more weeks of development, leading to a major downward correction of your individual mark.  Secondly, you will find it difficult to catch up with the progress of your team.  Thirdly, your track record of poor productivity will cause your team mates to lose trust in you, and it will take some time to rebuild that trust.

Choosing a schedule in the registration form is to make a commitment to your future team.  To maximise your marks, you should the longest schedule where you will still be able to meet the associated commitments.

\paragraph{Schedule start dates}
\tfaq[1]{0mm}{Can we choose when the project starts and ends, provided the duration of the schedule is adhered to?}{Small~group~project}
It is essential that the whole team works on the project at the same time!  It is not ok for members of the same team to work at different periods of the Semester.  That would prevent collaborative quality control.  Such an approach would also leave a subgroup of the team with final responsibility for completing the project towards the end.  

The end date of the project is the same for all teams: i.e. the project submission deadline.  The project must take place during a continuous period that ends in the project deadline.  If your team follows the four week schedule, the project must start about four weeks before the deadline and continue up until the deadline.  If your team follows the three/two week schedule, the project must start about two or three weeks before the deadline and continue up until the deadline.  

Teams should not start development earlier than the project requires.  This would exclude team members who signed up to a shorter schedule in the expectation that that would give them time to prepare to engage with the project by a particular date.  Of course, a team can start development sooner than required if all team members \emph{unanimously} agree to do so.  Please bear in mind that it is advisable to have one or two team coordination meetings before the scheduled start of the project.  Only development should not start until the project start date.

If a team is awarded an extension as mitigation for an \ac{MCF}, the extension should not be used to delay the start time of the project.

\paragraph{Default schedule}
\tfaq[1]{0mm}{What is the ``normal'' schedule?}{Small~group~project}
You should pick the schedule that is right for your situation, just prior to reading week.  Ideally, everyone would follow the six week schedule.  In practice, a substantial proportion of the class is unable to meet the obligations associated with that schedule for a variety of reasons.  Some students are behind on the independent learning activities associated with this module.  Some students are too busy working on coursework for other modules to engage with the small group project.  If you pick an overly demanding schedule, you will probably join a team that produce a better result overall.  However, individually you are likely to end up with a worse mark and you will end up frustrating and alienating fellow students.

\paragraph{Changing schedules}
\tfaq[1]{0mm}{I am no longer able to meet the expectations of the schedule I have chosen.  Can I change schedule or team?}{Small~group~project}
Before the registration deadline, you can submit the registration form repeatedly.  After the submission deadline, the information included in your final submission will be used to allocate you to a team.  Everyone in the team must follow the same schedule.  In other words, you cannot unilaterally decide to deviate from your chosen schedule because of new circumstances.  Team changes will not be possible either.

If you are affected by extenuating circumstances beyond your control, please used the College's mitigating circumstances process to address this.  For more information, please see Section~\ref{sect:mitigating-circumstances}.

\section{Assignment}
\index{small~group~project!assignment}
\index{assignment!small~group~project}
The small group project assignment will be released when the project starts.  You will be tasked with developing an information or content management system with Django.  In essence, such a system consists of a database, application logic implementing \ac{CRUD} operations, and a web based \ac{UI} to interact with the \ac{CRUD} operations.
\marginnote{\newthought{Further reading}: The assignment will be released on KEATS when the project starts.  Navigate to \emph{Small group project $\rightarrow$ Handbook} to find it.}

The assignment will be described in the form of \emph{objectives} and \emph{priorities}, \emph{not} as \emph{requirements}.  Objectives\index{objectives} describe what a client or a business needs -- what is of value to them.  Requirements\index{requirements} describe the features the software needs to provide.  As the assignment provides objectives, part of the task is to convert the objectives into requirements.

\marginnote{\newthought{Expectation}: \index{expectations!scope}Teams are \emph{not} expected to try and meet all objectives.  The project scope should be a conscious decision by the team based on the team's effective size\index{team~size!effective}, the team's ability and productivity, and expectations other that software functionality that the team should meet. (see Team marking scheme for more details).}
Teams decide the scope of what they will produce.  Team may decide to implement more or less functionality depending on the team's specific circumstances, and standard of work they can deliver.  A range of other aspects of the team's work are also assessed, including design, code quality, version control, deployment, and certain aspects of project and team management.  Certain improvements in the application's functionality will not increase your team's marks unless those other criteria are addressed.

\section{Technology and tools}
\subsection{Technology constraints}
\tfaq[1]{0mm}{What languages, frameworks, and technologies are allowed in the small group project?}{Small~group~project}
\index{small~group~project!technology~constraints}
\index{technology~constraints!small~group~project}
This project must be developed entirely with Python, Django, \ac{HTML}, \ac{CSS}, and Bootstrap.  You can install any Python/Django packages, as long as these can be installed through PIP as needed.  These must be recorded, with the versions you use, in the \texttt{requirements.txt} file.  Even though the tools you are using here do rely on other languages/technologies, your team must not produce code in any other languages or technologies, including Javascript/Typescript.  Using Bootstrap components that rely on Javascript are fine, but you should produce your own Javascript in the small group project.

If you have experience with web development, you will know that this makes some tasks a little harder.  Some problems are solved more easily and effectively with front-end languages.  However, these technology constraints aim to ensure most of what you need for the project is learned through the independent study component.  As no team member can be excluded from the project due to technology choices, it makes the project more inclusive.  Therefore, these technology constraints will be enforced.

Some \emph{starter code}\index{small~group~project!starter~code} will be released with the small group project assignment.  This starter code provides some initial pages, a User model and corresponding database migration, features to log in, log out, and constraint access based on log in/out status, database seeder code, and extensive test code.  You must extend your application from this starter code.  The starter code does not contribute towards the assessment.  For example, the automated tests included with the starter code are not counted towards testing marking criteria and only the additional tests you add are considered.

\subsection{Tools}
\tfaq[1]{0mm}{Which development tools should be used in the small group project?}{Small~group~project}
In the small group project, the team must use the following tools:
\begin{itemize}
\item Team Feedback: The team must use Team Feedback as outlined in Section~\ref{sect:tools:team-feedback}.  Pay particular attention to ensure that meeting attendance records are accurate, minutes are recorded in detail, collaborative coding sessions are recorded by the committer, and code contribution statistics are correct.  At the end of the project, Team Feedback is used to administer the peer assessment exercise.
\item Git and GitHub version control: your code must be produced and shared via a single git repository.  This repository must be shared via Team Feedback.
\item Trello: your team must maintain development tasks via a single Trello Kanban board.  This board must be shared via Team Feedback.
\item The code must be derived from the starting code.  Make sure that all build automation tools work specified in the original \texttt{README.md} file.  Do not introduce new requirements for creating the database, seeding the database, running the code in development, running the tests, and produce code coverage statistics.
\end{itemize}


\section{Project management approach}
\tfaq[1]{0mm}{How should the small group project be project managed?}{Small~group~project}
\index{small~group~project!project~management~approach}
\index{project~management~approach!small~group~project}
The Small Group Project Handbook outlines a detailed project management approach, including meeting agendas and meeting templates.  This approach is based heavily on the project management approach we discuss in the lectures.  Teams are expected to follow this approach carefully.  To ensure that that teams follow this approach, some observable features of the approach are considered in the assessment criteria.  However, the main reason why teams should adopt the specified approach is to promote team productive and manage risk.

\section{Deliverables}
\tfaq[1]{0mm}{What are the deliverables for the small group project?}{Small~group~project}
\index{small~group~project!deliverables}
\index{deliverables!small~group~project}
\marginnote{\newthought{Further reading}: On KEATS, navigate to \emph{Small group project $\rightarrow$ Handbook} to find a detailed description of the project deliverables.}
The team's deliverables at the end of the project consists of three parts:
\begin{enumerate}
\item A KEATS submission: On KEATS, you must submit a single ZIP file containing everything needed to install, configure, run, test, and evaluate the application.  This submission must include a \texttt{README.md} file, the source code, the database migrations, the test code, the \texttt{requirements.txt} file, and a self-assessment report named \texttt{self-assessment.pdf}.  To produce the self-assessment report, teams will be given a \LaTeX\index{latex@\LaTeX} document to fill out and compile.  Take care \emph{not} to include the git repository, the virtual environment, or the sqlite database file.
\item A deployed system: The web application must run on a production server accessible through a publicly available URL.  The URL must be provided in the README.md file.  The deployed system's database must be seeded with a substantial amount of data.  The system must be accessible with username/email and password combinations provided with the assignment.
\item Team management data on Team Feedback: Team Feedback must have access to your team's Git repository and Trello board.  All team meetings must have been minuted within 24 hours of each meeting taking place.
\end{enumerate}

\section{Assessment}
\subsection{Team marking criteria}
\tfaq[1]{0mm}{What are the criteria used to mark the team in the small group project, and how does the team's effective size affect they way these criteria are interpreted?}{Small~group~project}
The team deliverables of the small group project are primarily assessed on a range of criteria.  The most important are functionality, design, code, and testing.  However, there are other criteria as well that assess specific aspects of your submission.  The functionality criterion refer to the scope of the software: the range of features that were implemented, how well they focus on project objectives, how consistent their \ac{UI}, how polished their implementation is, and the overall ambition and technical achievement of the team's work.  Functionality is assessed in relation to the teams effective size\index{team~size!effective}.  The design and code criterion refer to the quality of your source code and the overall design of your system.  The expectations are based on Martin's books on ``clean code'' \citep{Martin:2009} and software architecture \citep{Martin:2017}.  The testing criterion assesses the coverage, comprehensiveness, and depth of your automated test suites.

\subsection{Team marking scheme}
\tfaq[1]{0mm}{How is the team mark for the small group project decided?}{Small~group~project}
\index{small~group~project!team~marking~scheme}
\index{team~marking~scheme!small~group~project}

\newcommand{\critical}{critical}
\newcommand{\necessary}{necessary}
\newcommand{\criterion}[5]{\emph{#1/#2}\if\relax\detokenize{#3}\relax\else{~(#3 marks)}\fi{: #4}\if\relax\detokenize{#5}\relax\else{\\\small\linespread{0.5}\emph{Hint}: #5}\fi}

The team marking scheme consists of a set of marking criteria, organised in 20 percentage point bands.  They are labelled, from lowest band to highest: I to V.
Each marking criterion is labelled to be either \critical or \necessary.  The criteria of any band above 20\% are only considered if \emph{all} the \critical criteria of the bands below it are met.  In other words, if a submission fails to meet a \critical criterion in a particular band, marking will be constrained by that band.  Therefore, it is important to consider all marking criteria, especially those in lower bands.  A \necessary criterion should be met as failing to meet them will affect the team's mark.  However, failing to meet a \necessary criterion will not prevent the team from meeting higher band criteria.  All band IV and V criteria are \critical.  Criterion VI.0 requires that all \necessary criteria are met.  Therefore, teams that fail to meet a \necessary criterion cannot attain band V.

Each marking criterion is assigned a certain amount of marks, adding up to 100\%.  If a criterion is met, the associated marks are awarded.  If a submission fails to meet a criterion, no marks are assigned for that criterion.

\paragraph{Band I: basic project requirements (0--20\%)}
\tfaq[1]{0mm}{What criteria must be met in order to attain a team mark in the 0--20\% range (band I) in the small group project?}{Small~group~project}
This band contains the essential requirements for a basic submission that includes minimal working software and everything the markers need to assess the team's work.  
\begin{itemize}[align=left, labelwidth=2.5em, labelsep=1em, leftmargin=3.5em]
\item[I.1]\criterion{Functionality}{\critical}{6}{The team has delivered a Django web application that capable of displaying pages with content (static or dynamic).  There are enough such pages relative to the team's effective size.  The application also delivers some content dynamically retrieved from a database.  The scope of this work is sufficient in scale for the given team size.}{}
\item[I.2]\criterion{Design}{\critical}{4}{The source code includes working models, views, and templates.}{}
\item[I.3]\criterion{Version control}{\critical}{4}{The marking team has access to the project's development history through a git repository shared via Team Feedback. It consists of mostly small commits throughout.}{Break down each development task you are assigned into smaller coding sub-tasks that you can tackle in a short period of time (e.g. 15--30 minutes).  Commit each solution to a subtask right away with an informative message.  Always push your local repository immediately after committing.}
\item[I.4]\criterion{Project management}{\necessary}{3}{Team Feedback contains an accurate record of the team's meetings.  This should include a record of at least one meeting per week during the team's development period, with an accurate attendance record.}{Assign one person in the team the responsibility to record each meeting on a laptop during the team's meeting.  Always record attendance accurately.}
\item[I.5]\criterion{Delivery}{\necessary}{3}{The team's submission meets the directory structure and file/directory naming requirements to the letter.  All required content is included.  The \texttt{README.md} file and the self assessment document have been filled out in full.}{Prepare your submission collectively as a team.  Read the instructions carefully.  Have one person check the work of another!  Make sure the code can still be installed, seeded, tested, and run from the submission file.}
\end{itemize}

\paragraph{Band II: requirements for an adequate team submission (20--40\%)}
\tfaq[1]{0mm}{What criteria must be met in order to attain a team mark in the 20--40\% range (band II) in the small group project?}{Small~group~project}
The criteria of this band are assessed \emph{only if all \critical} requirements of band I are met in full.  In combination with the band I, the criteria of this band reflect everything the team must achieve to attain a basic pass of the project.
\begin{itemize}[align=left, labelwidth=2em, labelsep=1em, leftmargin=3em]
\item[II.1]\criterion{Functionality}{\critical}{6}{The application enables users to create, read, update, and delete new data, using forms as necessary.  Such features are combined into tools that are useful for some of the intended end-users.  The scope and scale of the application is adequate for the team's effective size.}{}
\item[II.2]\criterion{Design}{\critical}{3}{The software uses Django forms to generate a forms (not raw HTML).}{}
\item[II.3]\criterion{Code}{\critical}{3}{The code is free from significant defects or major bugs.}{}
\item[II.4]\criterion{Testing}{\critical}{3}{The source code includes some new automated tests, at least two per effective team member.}{}
\item[II.5]\criterion{Delivery}{\necessary}{5}{The submission is evaluated with a specific set of build automation commands (specified with the assignment) to install a virtual environment, install required packages, setup and seed/unseed the database, run the tests, generate a code coverage report, and run the development server.  The markers will log into the server with predefined user access credentials.  These must be available to the markers exactly as specified so that they can assess the team's work \emph{without} (!) having to review the README.md file, analyse the code, or perform minor debugging.}{Prepare your submission collectively as a team.  Double or triple check: have one person check the work of another!}
\end{itemize}

\paragraph{Band III: requirements for a fair team submission (40--60\%)}
\tfaq[1]{0mm}{What criteria must be met in order to attain a team mark in the 40--60\% range (band III) in the small group project?}{Small~group~project}
The criteria of this band are assessed \emph{only if all \critical} requirements of the lower bands (bands I and II) are met in full.  In combination with the preceding bands, the criteria of this band reflect everything the team must achieve to attain a fair grade for the project.
\begin{itemize}[align=left, labelwidth=2em, labelsep=1em, leftmargin=3em]
\item[III.1]\criterion{Functionality}{\critical}{3}{The application possesses a set of working features that is moderately ambitious, given the team's effective size.}{}
\item[III.2]\criterion{Management}{\critical}{3}{The application's feature set is largely focussed on supporting a cohesive set of project objectives.  Isolated features that do not contribute to a fully supported/implemented objective are largely avoided.}{Achieving this criterion requires careful task allocation.  Define tasks that meet the INVEST criteria and only ever assign tasks that add value now.}
\item[III.3]\criterion{Design}{\critical}{2}{The software employs the Django framework effectively, making good use of models, views, templates, forms, and other components.  To meet this criterion, the aforementioned components must be present in the source code and used for the purpose Django intended.}{}
\item[III.4]\criterion{Code}{\critical}{3}{The code is reasonably clean.  Variables, functions, methods, and classes mostly have suitable names.  The code layout is mostly consistent.  Whitespace is used mostly consistently to separate functions, methods, classes, and other components.  Excessive/unnecessary whitespace is mostly avoided.  The code base mostly uses the same indentation symbol.}{Small lapses of judgement will be tolerated in this band, but make an effort.  Unless everyone's code is inspected by someone else before merging it with the main, it is impossible to enforce a basic level of code cleanliness.}
\item[III.5]\criterion{Testing}{\critical}{3}{The software comes with a test suite with good statement and branch coverage.  Failed tests are largely avoided.  Normally, each individual module should achieve at least 70\% statement coverage, and overall, 90\% of tests should pass.}{Each developer should write a test suite for their own source code as and when they write that source code.  Do not postpone writing tests.  Before merging a branch, inspect the test coverage of new source code.  Never merge a branch that causes tests to fail into the main branch until the problem is resolved. Never ever delegate test writing to give a previously disengaged team member in order to give them something to do: this is a recipe for failing this criterion.}
\item[III.6]\criterion{Deployment}{\critical}{2}{All the application's features work in both the development and the deployed version of the software.   The features are accessible via the user credentials specified in the assignment.  There are no features that have not been deployed.  The deployed version of the software is seeded with a substantial database, containing a sufficient volume records to perceive the application at scale.}{Start deploying a version of the application early and keep it updated.  Use a database seeder to seed the database as this will avoid problems.  Require every member of the team to test the deployed version immediately.  You cannot blame failing this criterion on another member of the team: the whole team is jointly responsible for meeting it.}
\item[III.7]\criterion{Code (comments)}{\necessary}{2}{The source code documents all (public) classes, methods, and functions that may be called from other modules.  All noise comments, including ``To Do'' comments and commented-out code, have been removed.}{Employ code inspections.  Require comments to be ``clean'' before merging a branch with the main.}
\item[III.8]\criterion{Code (file structure)}{\necessary}{2}{The source code is organised into a sensible file structure that meets Django and Python conventions.  Excessively large files and directories containing too many items at the same level are large avoided.  For the purposes of marking, the maximum number of lines in a source code file is set to 400, and the maximum number of subdirectories and file in a directory is set to 30.}{Employ code inspections.  Follow the naming/structure conventions used in the teaching materials.  Aim to keep file and directory sizes well below the limits.}
\end{itemize}

\paragraph{Band IV: requirements for a (very) good team submission (60--80\%)}
\tfaq[1]{0mm}{What criteria must be met in order to attain a team mark in the 60--80\% range (band IV) in the small group project?}{Small~group~project}
The criteria of this band are assessed \emph{only if all \critical} requirements of the lower bands (bands I--III) are met in full.  In combination with the preceding bands, the criteria of this band reflect everything the team must achieve to attain a good or very good grade for the project.
\begin{itemize}[align=left, labelwidth=2em, labelsep=1em, leftmargin=3em]
\item[IV.0]\criterion{Delivery}{\critical}{0}{All \necessary~criteria mentioned above are met.  No marks are assigned to this criterion.  However, if a submission fails to meet a \necessary criterion, the team's mark is capped at this band.}{}
\item[IV.1]\criterion{Functionality}{\critical}{3}{The application can be used to achieve an ambitious range of objectives.  The application's feature set is largely focussed on supporting a cohesive set of project objectives.  Isolated features that do not contribute to a fully supported/implemented objective are largely avoided.}{}
\item[IV.2]\criterion{Functionality}{\critical}{3}{Features are fully developed, offering an intuitive and flexible interface to end users.  Expectations include (but are not limited to) the following.  Dates/times are entered in an intuitive format consistent with UK conventions.  Individual records can be identified or selected without using data not intended to be used by users (e.g. primary keys, unless these have a special meaning).  Lists come with a range of facilities to navigate them, including pagination, ordering, and searching.}{When developing a new feature, start by ignoring this criterion and produce a basic working version.  Then, gradually add new tasks to the backlog to incrementally refine existing features.  Do not forget that the new code associated with this criterion requires cleanup and refactoring!  This takes time.}
\item[IV.3]\criterion{Management}{\critical}{2}{The application's user interface is consistent throughout.  Information screens, forms, and lists have the same look and feel.  The language/terminology used throughout the user interface is consistent.  Similar features have the same functionality set.  Where different screens possess equivalent controls and information, these can be found in the same place with the same look and feel.}{The level of coordination required to achieve this marking criterion goes substantially beyond that of the previous band.  The team will need discuss, agree, and document the organisation, look, and feel of the application and specific types of screens, possibly at multiple stages in the project.  User interface inspections will need to be incorporated into the team's task review processes before a task can be deemed completed.  This coordination tends to require time and, therefore, a longer schedule.}
\item[IV.4]\criterion{Design}{\critical}{3}{Views are critical and highly connected modules in a Django applications.  The bodies of view functions and methods are small and restricted to control logic.  Repetition of policies required by multiple views is prevented through effective reuse of code.}{This criterion requires systematic}
\item[IV.5]\criterion{Code (Source code)}{\critical}{3}{The source code meets high code cleanliness standards.  Variables, functions, methods, and classes \emph{consistently} have clear and descriptive names.  Code layout is consistent \emph{throughout}.  There are no long functions or methods: no method of function body consists of more than 25 lines in marking.  No function or method body has more than 2 levels of nesting.  The code is mostly DRY as significant code repetition has been avoided.}{To meet this criterion, a team's quality control standards need to be substantially more rigorous than those required to attain band D/C criteria.  Document your code inspections.  Auditing code inspections (i.e. reviewing whether inspectors spot all problems) can be helpful here.  Identifying repetitive code normally requires code reviews.}
\item[IV.6]\criterion{Code (Templates)}{\critical}{2}{High code cleanliness standards apply to templates.  Repetitive code in templates is avoided through effective use of template inheritance and template partials.  The templates contain no manual styling: CSS classes are used instead as necessary.}{Identifying repetitive code will require a systematic code review.  Issues such as local style attributes can be identified in an adequately organised inspection.}
\item[IV.7]\criterion{Testing}{\critical}{4}{The software comes with a test suite with impeccable statement and branch coverage.  All tests pass.}{Before merging a branch, inspect the test coverage of new source code.  Set stringent conditions on test coverage.  Do not merge a branch if it causes some tests to fail.  Consider adopting GitHub Actions to run the test suite.}
\end{itemize}

\paragraph{Band V: requirements for an exceptional team submission (80--100\%)}
\tfaq[1]{0mm}{What criteria must be met in order to attain a team mark in the 80--100\% range (band V) in the small group project?}{Small~group~project}
The criteria of this band are assessed \emph{only if all} requirements of the lower bands (bands I--IV) are met in full.  In combination with the preceding bands, the criteria of this band reflect everything the team must achieve to attain an exceptional grade for the project.
\begin{itemize}[align=left, labelwidth=2em, labelsep=1em, leftmargin=3em]
\item[V.1]\criterion{Functionality}{\critical}{5}{The team delivered an application that is very ambitious in scope and exceptionally polished, given the team's effective size.}{}
\item[V.2]\criterion{Management}{\critical}{3}{The team's time management has been excellent.  The final three days of the project were free from significant development activity (as manifested by code activity statistics).  All development and almost all refactoring took place before this period.  This has allowed the team to focus the final days of the project on quality assurance of the submission.}{}
\item[V.3]\criterion{Design}{\critical}{3}{The design achieves high cohesion and low coupling throughout.  Classes have limited responsibility, ideally a single responsibility.  Functions and methods do one thing only.}{}
\item[V.4]\criterion{Code (Source code)}{\critical}{2}{The source code meets exemplary code cleanliness standards.  Naming is consistent throughout the application.  All names make meaningful distinctions.  Function and method are extremely short with no more than 15 lines of code and 1 level of nesting.  The code includes no repetition.}{}
\item[V.5]\criterion{Code (Test code)}{\critical}{2}{High code cleanliness standards extend to the test code.  Test code uses clear, descriptive, and consistent names.  Test code repetition is minimal.  The bodies of test functions/methods are limited to 25 lines and 2 levels of nesting.}{}
\item[V.6]\criterion{Testing}{\critical}{5}{Spot inspections of test code shows that test suites have been carefully designed to ensure good coverage of input and output partitions, as well as potential causes of errors.}{This criterion is not concerned with code coverage, but with the range of test cases.  To meet this criterion, code inspection must examine test code as well as source code.}
\end{itemize}

Please bear in mind that, at level 5 (Year 2) of the course, marking criteria are more stringent than at level 4 (Year 1).  Attaining band V criteria is intended to be challenging.  While not impossible, it is far from the norm!


\subsection{Major mark correction}
\label{sect:sgp:major-mark-correction}
\tfaq[1]{0mm}{Under what conditions would my individual small group project mark be a severe reduction on the team's mark?}{Small~group~project}
\index{small~group~project!major~mark~correction}
\index{major~mark~correction!small~group~project}
As explained on page \pageref{individual-marking:reality-check}, it is not possible for the markers to make a precise, objective, and accurate assessments of the extent to which each individual in a team meets the individual expectations.  In the small group project -- a project where features are implemented incrementally/iteratively, and vertically -- team members should be engaged in similar coding tasks.  The mark scheme seeks to encourage all team members to attend all team meetings, commit to some work at each meeting, complete their task between meetings, and deliver that work at the meeting succeeding the one where they committed to the task.

\marginnote{\newthought{Expectation}: The code contribution statistics are derived from the team's shared repository.  Every student is responsible to ensure that their workstation\index{workstation!git} is set up correctly to ensure that their commits are associated with the correct GitHub account.  When coding, every student is responsible to ensure that they commit their work and push their work to the shared repository.  Use Team Feedback's features to correct attribution errors (see \emph{Team Feedback > Help > Git troubleshooting} for more help.}
To check to what extent that commitment was met, a relatively rudimentary assessment will be made based on a set of metrics:
\begin{enumerate}
\item\label{sgp:mmc:code-production} The number of weeks in which the number of committed line changes exceeds a minimum threshold.  This (admittedly crude) metric seeks to assess that an effort has been made to produce work.
\item\label{sgp:mmc:code-contribution} The number of weeks in which the number of line changes committed to the main branch exceeds a minimum threshold.  After commits are merged with the main branch, the line changes in those commits count as changes committed to the main branch, and the commits are attributed to the week the commit was made originally.  This crude metric assesses the amount of work the team could actually use, on account of it being merged with the main.
\item\label{sgp:mmc:meeting-attendance} The number of meetings attended in full (i.e. where the team member was present on-time, or 5 minutes late).
\end{enumerate}

\marginnote{\newthought{Expectation}: Line counts are not a good way to measure code contribution.  However, the thresholds are set low.  It is assumed that everyone builds/changes the models, views, templates, and test code for the features they are responsible for (as prescribed in the small group project approach).  This should make it trivial for a fully engaged member to meet the thresholds.}
Table~\ref{tab:small-group-project:major-correction:code-correction} summarises how metrics \ref{sgp:mmc:code-production} and \ref{sgp:mmc:code-production} are used to compute a major mark correction based on code contribution.  In this table, the number $n$ corresponds to the number of weeks that at least one member of the team made a full contribution to the code.  The thresholds for the number of committed line changes and the number of line changes committed to the main branch, referred to in metrics \ref{sgp:mmc:code-production} and \ref{sgp:mmc:code-production} respectively are defined in the Small Group Project handbook.

\begin{table}[ht]
\checkoddpage \ifoddpage \forcerectofloat \else \forceversofloat \fi
  \centering
  \fontfamily{ppl}\selectfont
  \begin{tabular}{p{30mm} p{30mm} p{30mm}}
    \toprule
    \#Weeks of minimum code contribution & \#Weeks of used minimum code contribution & Mark correction\\
    \midrule
    \multirow{4}{*}{$n-1$ or $n$} & 3 or more & no effect\\
    & 2 & -10\%\\
    & 1 & -20\%\\
    & 0 & -40\%\\
    \midrule
    \multirow{4}{*}{$n-2$} & 3 or more & -10\%\\
    & 2 & -20\%\\
    & 1 & -30\%\\
    & 0 & -50\%\\
    \midrule
    \multirow{4}{*}{$n-3$} & 3 or more & -25\%\\
    & 2 & -35\%\\
    & 1 & -45\%\\
    & 0 & -65\%\\
    \midrule
    \multirow{3}{*}{$n-4$} & 2 & -50\%\\
    & 1 & -60\%\\
    & 0 & -80\%\\
    \midrule
    \multirow{2}{*}{$n-5$} & 1 & -70\%\\
    & 0 & -90\%\\
    \midrule
    0 & 0 & -95\%\\
    \bottomrule
  \end{tabular}
  \caption{Major individual mark correction for code contribution in teams with a development period of $n$ weeks.  Weeks are defined as the Monday-Sunday intervals between the start and end of the project.}
  \label{tab:small-group-project:major-correction:code-correction}
  %\zsavepos{pos:normaltab}
\end{table}

Table~\ref{tab:small-group-project:major-correction:attendance} summarises how metric \ref{sgp:mmc:meeting-attendance} is used to compute a major mark correction based on meeting attendance.  Please note that team members ought to attend all meetings: 80\% is \emph{not} ``enough''.

\begin{table}[ht]
\checkoddpage \ifoddpage \forcerectofloat \else \forceversofloat \fi
  \centering
  \fontfamily{ppl}\selectfont
  \begin{tabular}{p{60mm} p{30mm}}
    \toprule
    \% of meetings where team member is ``present'' or ``5 minutes late'' & Mark correction\\
    \midrule
    80\% or above & no effect\\
    60\% -- 79\% & -20\%\\
    40\% -- 59\% & -50\%\\
    20\% -- 39\% & -80\%\\
    less than 20\% & -100\%\\
    \bottomrule
  \end{tabular}
  \caption{Major individual mark correction for meeting attendance.}
  \label{tab:small-group-project:major-correction:attendance}
  %\zsavepos{pos:normaltab}
\end{table}

The largest absolute value of the two major correction is applied as the major mark correction for individual marking purposes.  Most students normally do not get a major mark correction from the team mark.  Where major mark corrections are applied, the peer assessments are reviewed manually to review whether there is a reason not to apply the major correction. 


\subsection{Minor mark redistribution}
\label{sect:sgp:minor-mark-redistribution}
\tfaq[1]{0mm}{If my individual mark is not severely reduced, do I receive the same mark as my team members in the small group project?}{Small~group~project}
\index{small~group~project!minor~mark~redistribution}
\index{minor~mark~redistribution!small~group~project}
Only students whose mark is unaffected by a major mark correction are considered for minor mark redistribution.  This section outlines briefly how the policy discussed on page \pageref{minor-mark-redistribution} is applied to the small group project.

After the small group project, you will participate in a peer assessment exercise.  The minor mark adjustment is primarily based on the peer assessments.  Only peer assessments about students whose mark is not subjected to a major mark correction are considered.

The peer assessment exercise includes a set of multiple-choice questions for each of your peers.  Scores are associated with each of the answers.  A weighted sum of these scores is calculated to produce a score for each peer assessment by a reviewer about a reviewee.  The weights favour engagement and contribution, but no question is weightless.  The peer assessment scores for each reviewer are normalised so that each reviewer within a team produces the same average peer assessment scores across the reviewees considered for minor mark redistribution.  The peer assessment score for a student is the average of the normalised peer assessment scores they received.  

\begin{table}[ht]
\checkoddpage \ifoddpage \forcerectofloat \else \forceversofloat \fi
  \centering
  \fontfamily{ppl}\selectfont
  \begin{tabular}{p{30mm} p{30mm} p{40mm}}
    \toprule
    Peer assessment score & Code contribution statistics & Mark redistribution\\
    \midrule
    Above average by a substantial margin & Above average & +3, +4, or +5 depending on the peer assessment score\\
    Below average by a substantial margin & Below average & -3, -4, or -5 depending on the peer assessment score\\
    \multicolumn{2}{p{60mm}}{Any other combination} & between -2 and -2 (inclusive), as necessary to ensure the sum of the adjustments is zero\\ 
    \bottomrule
  \end{tabular}
  \caption{Summary of the minor mark redistribution policy in the small group project}
  \label{tab:small-group-project:minor-mark-redistribution}
  %\zsavepos{pos:normaltab}
\end{table}

Table~\ref{tab:small-group-project:minor-mark-redistribution} summarises how minor corrections are decided.  In essence, team members with scores substantially above or below average, and with code contributions that above or below average respectively, are adjusted by a mark of at least 3 and at most 5.  The amount depends on the difference between the peer assessment score and the average peer assessment score.  All other individual are adjusted up or down by at most 2 to ensure that the sum of all corrections is 0.  

The peer assessments manually reviewed to check for anomalies, such as cases where individuals are dishonest and try to game the system.

\subsection{Peer assessments}
\tfaq[1]{0mm}{How are the peer assessments marked in the small group project?}{Small~group~project}
\index{small~group~project!peer~assessment~mark}
\index{peer~assessments!small~group~project}
The peer assessments you write about your team mates are marked directly, focussing on the open-ended feedback text that you write.  This contributes towards 5\% of the small group project mark.  The following assessment criteria are used to produce this mark:
\begin{itemize}[align=left, labelwidth=2.5em, labelsep=1em, leftmargin=3.5em]
\item[0\%]  The feedback text contains no meaningful write-up.
\item[20\%]  A peer assessment was submitted for almost every team member.  The feedback text of each submitted peer assessment contains a single sentence explaining the scores for that peer assessment.
\item[40\%]  A peer assessment was submitted for every team member.  Each peer assessment includes at least a 2--3 sentence explanation for the peer assessment scores.  
\item[60\%]  A peer assessment was submitted for every team member.  Each peer assessment includes at least a 2--3 sentence explanation for the peer assessment scores, recognising strengths and weaknesses, as well as some valid constructive feedback that, when acted on, would help the reviewee in future projects.   The text is largely free from grammatical errors.
\item[80\%]  A peer assessment was submitted for every team member.  Each peer assessment includes a thorough explanation for the peer assessment scores,  recognising strengths and weaknesses, as well as valid constructive feedback that, when acted on, would help the reviewee in future projects.  The peer assessment is written thoughtfully, and internally cohesive.  The constructive feedback logically follows from the explanation.  The text is free from grammatical errors.
\item[100\%]  An unusually extensive peer assessment was submitted for every team member.  Each peer assessment includes a thorough explanation for the peer assessment scores,  recognising strengths and weaknesses, as well as valid constructive feedback that, when acted on, would help the reviewee in future projects.  The peer assessment is written thoughtfully, and internally cohesive.  The constructive feedback logically follows from the explanation.  The ideas are rooted in project management theory, and sometimes insightful.  The text is free from grammatical errors. 
\end{itemize}

In some teams, you may have a team member who failed to engage with the team at all: i.e. they did not attend any meetings or produce any work.  Such a person may remain in your team if they signed up to participate in the small group project and did not defer early on.  Obviously, there is little to write about in such a situation.  The feedback to write for such a peer can be limited to a single sentence explaining the lack of engagement without affecting your peer assessment mark.
If a person's engagement was very limited, but they attended some meetings or produced some work, focus on your interactions with them.