\chapter{Git Command Quick Reference}

\noindent\textbf{Conventions.} Words in UPPERCASE (e.g., \texttt{FILE}, \texttt{BRANCH}, \texttt{URL}) are placeholders.

\section{Repository Setup}

\begin{itemize}
  \item \texttt{git init} -- create a new repository in the current directory.
  \item \texttt{git clone URL} -- clone a remote repository.
\end{itemize}

\section{Inspecting State}

\begin{itemize}
  \item \texttt{git status} -- show changed, staged, and untracked files.
  \item \texttt{git log} -- show commit history.
  \item \texttt{git diff} -- show unstaged changes.
  \item \texttt{git diff --staged} -- show staged changes.
\end{itemize}

\section{Staging and Committing}

\begin{itemize}
  \item \texttt{git add -A} -- stage all modified and untracked files.
  \item \texttt{git commit -m "message"} -- commit staged changes.
\end{itemize}

\section{Branches}

\begin{itemize}
  \item \texttt{git branch} -- list branches.
  \item \texttt{git branch BRANCH} -- create a branch.
  \item \texttt{git checkout BRANCH} -- switch to a branch.
  \item \texttt{git merge BRANCH} -- merge into current branch.
\end{itemize}

\section{Working with Remotes}

\begin{itemize}
  \item \texttt{git remote -v} -- list remotes.
  \item \texttt{git pull} -- fetch and merge remote updates.
  \item \texttt{git push} -- upload local commits.
  \item \texttt{git push -u origin BRANCH} -- push the local branch named BRANCH
        to the remote repository called \texttt{origin}, and set that remote
        branch as the default "upstream" for the local branch. This means that
        in the future you can simply run \texttt{git push} or \texttt{git pull}
        without specifying the remote or branch name, because Git will remember
        the association.
\end{itemize}
