\chapter{Introduction}

\tfaq[1]{10mm}{What is the software engineering group project module?}{General~questions}

Congratulations on reaching Year 2 of your course and welcome to the Software Engineering Group Project module.

This document is a handbook on the \ac{SEG} module.  It identifies the objectives of the module, explains the way the module is organised, sets out the most important expectations, provides deadlines and key dates, and specifies how you will be assessed.  Formally,  \ac{SEG} is a \emph{core} module, which means that you must pass it (a fail mark cannot be compensated or condoned).  It contributes 30 credits and takes places during both Semesters 1 and 2 in Year 2.  It is assessed by two pieces of coursework, each based around a group project.

\section{Why a handbook?}

\tfaq[1]{0mm}{Why does this module have a handbook?}{Handbook}

The \ac{SEG} module differs considerably from other modules in that it concerns producing software systems \emph{with others in the context of a substantial project}.  Therefore, this module focusses on team-based software engineering, project management, and collaboration with others, and it is assessed almost exclusively via group work.  Compared to individual assessment, group work comes with considerable additional uncertainty or risk, because the achievements of your team are dependent on the abilities, attitudes, behaviours, and interactions of others, not just your own.  The module is designed around managing that risk, promoting (rewarding) constructive and sustained engagement with team mates, effective collaboration, and sound project and team management.  Learning and work practices that may have worked well for you in other modules, may not work for this module.  

This handbook describes in considerable detail what is expected of you in this module.  It describes the organisation of the module, what you need to learn to be successful in this module, the tools we use or can use, and how the assessments are organised and marked.  

\section{How to use this handbook}

\tfaq[1]{0mm}{How should the module handbook be used?}{Handbook}

This document has become rather long, so I do not expect you to sit down and read the whole thing as soon as possible.  The handbook has been written to provide relevant information at key stages of the year, and to be used as a reference text to enable you to look up information as and when you need it.

Some of the chapters are organised to be relevant to key stages of the module.  The Study Guide chapter (Chapter~\ref{ch:study-guide} should be read \emph{before we meet in the first lecture}.  This chapter tells you what this module is about, what teaching/support is provided, how you will be assessed, and what tools you need to acquire to engage with the module.  While the module will be introduced in the first lecture, I will not repeat everything in the Study Guide chapter as that would make for an overly information dense session.  If the Study Guide raises any questions, do make a note of them for the Q\&A component of the session.  

Modern software engineering is heavily reliant to tools.  In this module, you will learn how to use a range of software engineering tools and gain experience in using them in team-based projects.  The Tools chapter (Chapter~\ref{ch:tools}) presents of overview of the tools you require and what you need to use them for.  In order to stay on track with the module, it is crucial that you install certain tools before the second week of teaching.

Next, familiarise yourself gradually with the General Expectations chapter (Chapter~\ref{ch:expectations}) in the first weeks of the academic year.  This chapter will give you an idea of what is expected of you and your team mates in the remainder of the academic year.  This handbook includes chapters on Project Management (Chapter~\ref{ch:project-management}) and Software Quality Assurance (Chapter~\ref{ch:software-quality}).  These chapters provide more specific expectations on how you should be working in the group projects and what standard of work is expected of teams.  It is worth reviewing these during the first (small) group project you undertake as a guide.

You will be undertaking two group projects in this module: a Small Group Project and a Major Group Project.  The chapters by the same names (Chapters~\ref{ch:small-group-project} and \ref{ch:major-group-project} respectively) advise you how these projects are organised, and how you will be assessed.  You should review these chapters before the respective group projects assignments start.

This handbook is provided as a searchable PDF document, so that you can look up specific information as and when you need it.  In the margins of this text, you will find \acp{FAQ} that the main text answers alongside it.  The text is organised such that each lowest level section, subsection, or paragraph answers a \ac{FAQ}: a question that at least some students have each year.  You can find a list of the \acp{FAQ} (Chapter~\ref{ch:faq}) at the end of the handbook, along with the page number where that question and the corresponding answer appears in the handbook.  The handbook is also indexed with key terms, so that you can look up important terms and information.