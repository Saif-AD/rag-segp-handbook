\documentclass{article}

\usepackage[margin=2cm,a3paper]{geometry}
\usepackage{longtable}
\usepackage{makecell}

\renewcommand{\familydefault}{\sfdefault}

\title{Software Engineering Group Project: Independent Learning Schedule}
\date{}
\begin{document}
\maketitle
During the first six weeks of the Software Engineering Group Project module, all students are expected to complete a set of independent learning activities, in preparation for participation in the first group project coursework: the small group project.  The learning activities consist of videos, reading material, and exercises on three different subjects: Python/Django development, DevOps (including version control), and project management.  Everyone learns at a different speed and you should prioritise comprehension over speed.  To help you organise and plan your independent learning, this document proposes three schedules.  

You are advised to try and follow (or exceed!) the \emph{recommended schedule} if you can as this ensures you complete all independent learning by reading week and enables you to participate in the six week small group project schedule.  Most Year 2 students are capable of following this schedule.  However, quite a few will struggle to muster the discipline and good time management this schedule requires.  The \emph{slow schedule} spreads out the material a little bit, with a view to prepare you to participate in the four week small group project schedule.  The \emph{minimal schedule} specifies the slowest pace at which you can engage with the material, and prepares you for the three/two week small group project schedule.  

It is critical that you start independent learning as soon as possible and maintain a good pace in the early weeks.  Catching up with a faster schedule later in the term is rarely achievable as the workload for other modules tends to increase as the semester progresses.  If you fall behind the minimal schedule, it is important to discuss this with the module lecturer or your personal tutor.



\begin{longtable}[t]{|p{40mm}|p{65mm}|p{65mm}|p{65mm}|}
\hline
Week & Recommended schedule & Slow schedule & Minimal schedule\\
\hline
\endhead
\makecell[tl]{w/c 29 Sep 2025}
& \makecell[tl]{
\textbf{Python/Django training}\\
(1.1) Introduction to Python, \\
(1.2) Built-in data structures in Python, \\
Exercise 1.2, \\
(1.3) Structured programming in Python, \\
(1.4) Functions and procedures of Python, \\
Exercise 1.4, \\
(1.5) Exceptions in Python, \\
Exercise 1.5, \\
(1.6) OOP in Python, \\
Exercise 1.6, \\
(1.7) Modules \& packages in Python \\
\\
\textbf{Devops training}\\
\\
\textbf{Software projects training}\\
(1.1) Introduction to project management\\
(1.2) Project constraints\\
} & \makecell[tl]{
\textbf{Python/Django training}\\
\\
\textbf{Devops training}\\
\\
\textbf{Software projects training}\\
(1.1) Introduction to project management\\
(1.2) Project constraints\\
}
& \makecell[tl]{
\textbf{Python/Django training}\\
\\
\textbf{Devops training}\\
\\
\textbf{Software projects training}\\
(1.1) Introduction to project management\\
(1.2) Project constraints\\
}
\\\hline
\makecell[tl]{w/c 6 Oct 2025} 
& \makecell[tl]{
\textbf{Python/Django training}\\
\\
\textbf{Devops training}\\
\\
\textbf{Software projects training}\\
} & \makecell[tl]{
\textbf{Python/Django training}\\
\\
\textbf{Devops training}\\
\\
\textbf{Software projects training}\\
} & \makecell[tl]{
\textbf{Python/Django training}\\
\\
\textbf{Devops training}\\
\\
\textbf{Software projects training}\\
}
\\\hline
\makecell[tl]{w/c 13 Oct 2025} & & & \\\hline
\makecell[tl]{w/c 20 Oct 2025} & & & \\\hline
\makecell[tl]{w/c 27 Oct 2025} & & & \\\hline
\makecell[tl]{Reading week} & & & \\\hline
\makecell[tl]{w/c 10 Nov 2025} & & & \\\hline
\makecell[tl]{w/c 17 Nov 2025} & & & \\\hline
\makecell[tl]{w/c 24 Nov 2025} & & & \\\hline
\end{longtable}
\end{document}