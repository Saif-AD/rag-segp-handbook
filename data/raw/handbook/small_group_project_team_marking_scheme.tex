\tfaq[1]{0mm}{How is the team mark for the small group project decided?}{Small~group~project}
\index{small~group~project!team~marking~scheme}
\index{team~marking~scheme!small~group~project}

\newcommand{\critical}{critical}
\newcommand{\necessary}{necessary}
\newcommand{\criterion}[5]{\emph{#1/#2}\if\relax\detokenize{#3}\relax\else{~(#3 marks)}\fi{: #4}\if\relax\detokenize{#5}\relax\else{\\\small\linespread{0.5}\emph{Hint}: #5}\fi}

The team marking scheme consists of a set of marking criteria, organised in 20 percentage point bands.  They are labelled, from lowest band to highest: I to V.
Each marking criterion is labelled to be either \critical or \necessary.  The criteria of any band above 20\% are only considered if \emph{all} the \critical criteria of the bands below it are met.  In other words, if a submission fails to meet a \critical criterion in a particular band, marking will be constrained by that band.  Therefore, it is important to consider all marking criteria, especially those in lower bands.  A \necessary criterion should be met as failing to meet them will affect the team's mark.  However, failing to meet a \necessary criterion will not prevent the team from meeting higher band criteria.  All band IV and V criteria are \critical.  Criterion VI.0 requires that all \necessary criteria are met.  Therefore, teams that fail to meet a \necessary criterion cannot attain band V.

Each marking criterion is assigned a certain amount of marks, adding up to 100\%.  If a criterion is met, the associated marks are awarded.  If a submission fails to meet a criterion, no marks are assigned for that criterion.

\paragraph{Band I: basic project requirements (0--20\%)}
\tfaq[1]{0mm}{What criteria must be met in order to attain a team mark in the 0--20\% range (band I) in the small group project?}{Small~group~project}
This band contains the essential requirements for a basic submission that includes minimal working software and everything the markers need to assess the team's work.  
\begin{itemize}[align=left, labelwidth=2.5em, labelsep=1em, leftmargin=3.5em]
\item[I.1]\criterion{Functionality}{\critical}{6}{The team has delivered a Django web application that capable of displaying pages with content (static or dynamic).  There are enough such pages relative to the team's effective size.  The application also delivers some content dynamically retrieved from a database.  The scope of this work is sufficient in scale for the given team size.}{}
\item[I.2]\criterion{Design}{\critical}{4}{The source code includes working models, views, and templates.}{}
\item[I.3]\criterion{Version control}{\critical}{4}{The marking team has access to the project's development history through a git repository shared via Team Feedback. It consists of mostly small commits throughout.}{Break down each development task you are assigned into smaller coding sub-tasks that you can tackle in a short period of time (e.g. 15--30 minutes).  Commit each solution to a subtask right away with an informative message.  Always push your local repository immediately after committing.}
\item[I.4]\criterion{Project management}{\necessary}{3}{Team Feedback contains an accurate record of the team's meetings.  This should include a record of at least one meeting per week during the team's development period, with an accurate attendance record.}{Assign one person in the team the responsibility to record each meeting on a laptop during the team's meeting.  Always record attendance accurately.}
\item[I.5]\criterion{Delivery}{\necessary}{3}{The team's submission meets the directory structure and file/directory naming requirements to the letter.  All required content is included.  The \texttt{README.md} file and the self assessment document have been filled out in full.}{Prepare your submission collectively as a team.  Read the instructions carefully.  Have one person check the work of another!  Make sure the code can still be installed, seeded, tested, and run from the submission file.}
\end{itemize}

\paragraph{Band II: requirements for an adequate team submission (20--40\%)}
\tfaq[1]{0mm}{What criteria must be met in order to attain a team mark in the 20--40\% range (band II) in the small group project?}{Small~group~project}
The criteria of this band are assessed \emph{only if all \critical} requirements of band I are met in full.  In combination with the band I, the criteria of this band reflect everything the team must achieve to attain a basic pass of the project.
\begin{itemize}[align=left, labelwidth=2em, labelsep=1em, leftmargin=3em]
\item[II.1]\criterion{Functionality}{\critical}{6}{The application enables users to create, read, update, and delete new data, using forms as necessary.  Such features are combined into tools that are useful for some of the intended end-users.  The scope and scale of the application is adequate for the team's effective size.}{}
\item[II.2]\criterion{Design}{\critical}{3}{The software uses Django forms to generate a forms (not raw HTML).}{}
\item[II.3]\criterion{Code}{\critical}{3}{The code is free from significant defects or major bugs.}{}
\item[II.4]\criterion{Testing}{\critical}{3}{The source code includes some new automated tests, at least two per effective team member.}{}
\item[II.5]\criterion{Delivery}{\necessary}{5}{The submission is evaluated with a specific set of build automation commands (specified with the assignment) to install a virtual environment, install required packages, setup and seed/unseed the database, run the tests, generate a code coverage report, and run the development server.  The markers will log into the server with predefined user access credentials.  These must be available to the markers exactly as specified so that they can assess the team's work \emph{without} (!) having to review the README.md file, analyse the code, or perform minor debugging.}{Prepare your submission collectively as a team.  Double or triple check: have one person check the work of another!}
\end{itemize}

\paragraph{Band III: requirements for a fair team submission (40--60\%)}
\tfaq[1]{0mm}{What criteria must be met in order to attain a team mark in the 40--60\% range (band III) in the small group project?}{Small~group~project}
The criteria of this band are assessed \emph{only if all \critical} requirements of the lower bands (bands I and II) are met in full.  In combination with the preceding bands, the criteria of this band reflect everything the team must achieve to attain a fair grade for the project.
\begin{itemize}[align=left, labelwidth=2em, labelsep=1em, leftmargin=3em]
\item[III.1]\criterion{Functionality}{\critical}{3}{The application possesses a set of working features that is moderately ambitious, given the team's effective size.}{}
\item[III.2]\criterion{Management}{\critical}{3}{The application's feature set is largely focussed on supporting a cohesive set of project objectives.  Isolated features that do not contribute to a fully supported/implemented objective are largely avoided.}{Achieving this criterion requires careful task allocation.  Define tasks that meet the INVEST criteria and only ever assign tasks that add value now.}
\item[III.3]\criterion{Design}{\critical}{2}{The software employs the Django framework effectively, making good use of models, views, templates, forms, and other components.  To meet this criterion, the aforementioned components must be present in the source code and used for the purpose Django intended.}{}
\item[III.4]\criterion{Code}{\critical}{3}{The code is reasonably clean.  Variables, functions, methods, and classes mostly have suitable names.  The code layout is mostly consistent.  Whitespace is used mostly consistently to separate functions, methods, classes, and other components.  Excessive/unnecessary whitespace is mostly avoided.  The code base mostly uses the same indentation symbol.}{Small lapses of judgement will be tolerated in this band, but make an effort.  Unless everyone's code is inspected by someone else before merging it with the main, it is impossible to enforce a basic level of code cleanliness.}
\item[III.5]\criterion{Testing}{\critical}{3}{The software comes with a test suite with good statement and branch coverage.  Failed tests are largely avoided.  Normally, each individual module should achieve at least 70\% statement coverage, and overall, 90\% of tests should pass.}{Each developer should write a test suite for their own source code as and when they write that source code.  Do not postpone writing tests.  Before merging a branch, inspect the test coverage of new source code.  Never merge a branch that causes tests to fail into the main branch until the problem is resolved. Never ever delegate test writing to give a previously disengaged team member in order to give them something to do: this is a recipe for failing this criterion.}
\item[III.6]\criterion{Deployment}{\critical}{2}{All the application's features work in both the development and the deployed version of the software.   The features are accessible via the user credentials specified in the assignment.  There are no features that have not been deployed.  The deployed version of the software is seeded with a substantial database, containing a sufficient volume records to perceive the application at scale.}{Start deploying a version of the application early and keep it updated.  Use a database seeder to seed the database as this will avoid problems.  Require every member of the team to test the deployed version immediately.  You cannot blame failing this criterion on another member of the team: the whole team is jointly responsible for meeting it.}
\item[III.7]\criterion{Code (comments)}{\necessary}{2}{The source code documents all (public) classes, methods, and functions that may be called from other modules.  All noise comments, including ``To Do'' comments and commented-out code, have been removed.}{Employ code inspections.  Require comments to be ``clean'' before merging a branch with the main.}
\item[III.8]\criterion{Code (file structure)}{\necessary}{2}{The source code is organised into a sensible file structure that meets Django and Python conventions.  Excessively large files and directories containing too many items at the same level are large avoided.  For the purposes of marking, the maximum number of lines in a source code file is set to 400, and the maximum number of subdirectories and file in a directory is set to 30.}{Employ code inspections.  Follow the naming/structure conventions used in the teaching materials.  Aim to keep file and directory sizes well below the limits.}
\end{itemize}

\paragraph{Band IV: requirements for a (very) good team submission (60--80\%)}
\tfaq[1]{0mm}{What criteria must be met in order to attain a team mark in the 60--80\% range (band IV) in the small group project?}{Small~group~project}
The criteria of this band are assessed \emph{only if all \critical} requirements of the lower bands (bands I--III) are met in full.  In combination with the preceding bands, the criteria of this band reflect everything the team must achieve to attain a good or very good grade for the project.
\begin{itemize}[align=left, labelwidth=2em, labelsep=1em, leftmargin=3em]
\item[IV.0]\criterion{Delivery}{\critical}{0}{All \necessary~criteria mentioned above are met.  No marks are assigned to this criterion.  However, if a submission fails to meet a \necessary criterion, the team's mark is capped at this band.}{}
\item[IV.1]\criterion{Functionality}{\critical}{3}{The application can be used to achieve an ambitious range of objectives.  The application's feature set is largely focussed on supporting a cohesive set of project objectives.  Isolated features that do not contribute to a fully supported/implemented objective are largely avoided.}{}
\item[IV.2]\criterion{Functionality}{\critical}{3}{Features are fully developed, offering an intuitive and flexible interface to end users.  Expectations include (but are not limited to) the following.  Dates/times are entered in an intuitive format consistent with UK conventions.  Individual records can be identified or selected without using data not intended to be used by users (e.g. primary keys, unless these have a special meaning).  Lists come with a range of facilities to navigate them, including pagination, ordering, and searching.}{When developing a new feature, start by ignoring this criterion and produce a basic working version.  Then, gradually add new tasks to the backlog to incrementally refine existing features.  Do not forget that the new code associated with this criterion requires cleanup and refactoring!  This takes time.}
\item[IV.3]\criterion{Management}{\critical}{2}{The application's user interface is consistent throughout.  Information screens, forms, and lists have the same look and feel.  The language/terminology used throughout the user interface is consistent.  Similar features have the same functionality set.  Where different screens possess equivalent controls and information, these can be found in the same place with the same look and feel.}{The level of coordination required to achieve this marking criterion goes substantially beyond that of the previous band.  The team will need discuss, agree, and document the organisation, look, and feel of the application and specific types of screens, possibly at multiple stages in the project.  User interface inspections will need to be incorporated into the team's task review processes before a task can be deemed completed.  This coordination tends to require time and, therefore, a longer schedule.}
\item[IV.4]\criterion{Design}{\critical}{3}{Views are critical and highly connected modules in a Django applications.  The bodies of view functions and methods are small and restricted to control logic.  Repetition of policies required by multiple views is prevented through effective reuse of code.}{This criterion requires systematic}
\item[IV.5]\criterion{Code (Source code)}{\critical}{3}{The source code meets high code cleanliness standards.  Variables, functions, methods, and classes \emph{consistently} have clear and descriptive names.  Code layout is consistent \emph{throughout}.  There are no long functions or methods: no method of function body consists of more than 25 lines in marking.  No function or method body has more than 2 levels of nesting.  The code is mostly DRY as significant code repetition has been avoided.}{To meet this criterion, a team's quality control standards need to be substantially more rigorous than those required to attain band D/C criteria.  Document your code inspections.  Auditing code inspections (i.e. reviewing whether inspectors spot all problems) can be helpful here.  Identifying repetitive code normally requires code reviews.}
\item[IV.6]\criterion{Code (Templates)}{\critical}{2}{High code cleanliness standards apply to templates.  Repetitive code in templates is avoided through effective use of template inheritance and template partials.  The templates contain no manual styling: CSS classes are used instead as necessary.}{Identifying repetitive code will require a systematic code review.  Issues such as local style attributes can be identified in an adequately organised inspection.}
\item[IV.7]\criterion{Testing}{\critical}{4}{The software comes with a test suite with impeccable statement and branch coverage.  All tests pass.}{Before merging a branch, inspect the test coverage of new source code.  Set stringent conditions on test coverage.  Do not merge a branch if it causes some tests to fail.  Consider adopting GitHub Actions to run the test suite.}
\end{itemize}

\paragraph{Band V: requirements for an exceptional team submission (80--100\%)}
\tfaq[1]{0mm}{What criteria must be met in order to attain a team mark in the 80--100\% range (band V) in the small group project?}{Small~group~project}
The criteria of this band are assessed \emph{only if all} requirements of the lower bands (bands I--IV) are met in full.  In combination with the preceding bands, the criteria of this band reflect everything the team must achieve to attain an exceptional grade for the project.
\begin{itemize}[align=left, labelwidth=2em, labelsep=1em, leftmargin=3em]
\item[V.1]\criterion{Functionality}{\critical}{5}{The team delivered an application that is very ambitious in scope and exceptionally polished, given the team's effective size.}{}
\item[V.2]\criterion{Management}{\critical}{3}{The team's time management has been excellent.  The final three days of the project were free from significant development activity (as manifested by code activity statistics).  All development and almost all refactoring took place before this period.  This has allowed the team to focus the final days of the project on quality assurance of the submission.}{}
\item[V.3]\criterion{Design}{\critical}{3}{The design achieves high cohesion and low coupling throughout.  Classes have limited responsibility, ideally a single responsibility.  Functions and methods do one thing only.}{}
\item[V.4]\criterion{Code (Source code)}{\critical}{2}{The source code meets exemplary code cleanliness standards.  Naming is consistent throughout the application.  All names make meaningful distinctions.  Function and method are extremely short with no more than 15 lines of code and 1 level of nesting.  The code includes no repetition.}{}
\item[V.5]\criterion{Code (Test code)}{\critical}{2}{High code cleanliness standards extend to the test code.  Test code uses clear, descriptive, and consistent names.  Test code repetition is minimal.  The bodies of test functions/methods are limited to 25 lines and 2 levels of nesting.}{}
\item[V.6]\criterion{Testing}{\critical}{5}{Spot inspections of test code shows that test suites have been carefully designed to ensure good coverage of input and output partitions, as well as potential causes of errors.}{This criterion is not concerned with code coverage, but with the range of test cases.  To meet this criterion, code inspection must examine test code as well as source code.}
\end{itemize}

Please bear in mind that, at level 5 (Year 2) of the course, marking criteria are more stringent than at level 4 (Year 1).  Attaining band V criteria is intended to be challenging.  While not impossible, it is far from the norm!