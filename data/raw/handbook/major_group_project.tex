\chapter{Major group project}
\label{ch:major-group-project}
\section{Group allocation}
\tfaq[1]{0mm}{How are teams formed in the major group project?}{Major~group~project}
\index{group~allocation!major~group~project}
\index{major~group~project!group~allocation}
In the major group project, you will be working in teams of approximately 8 ($\pm 2$) people.  There will be a selection of projects and teams can choose their own technology stack.  As it is easier to do this in a team of people with shared project and technology preferences, students are strongly encouraged to form their own teams.    Students who are unable to find a team can register to be allocated to one, in a similar manner as for the small group project.  If you would like to work with some people, but are unable to create a group with 6--10 members, you can create a smaller project team to be extended through allocation.  Students in teams that came together with a shared understanding of the kind of project they want to do, and how they wish to work tend to have the best experience.

To find a major group project team, it helps to start networking as early as possible: from the start of the academic year.  With the team formation deadline still far away, you have the time to get to know fellow students, what their interests, talents, goals, and values are.  The lectures, small group tutorials, and the small group project all offer opportunities to meet and talk to fellow students.   While you want to work with people who share similar objectives, values, and project type and technology preferences, make sure to diversify your team.  Team tend to benefit from having people with different backgrounds, personalities, and talents.

To register a student-formed team, one person has to create the team on Team Feedback\index{Team~Feedback!team~formation}.  As team owner, they can then send invitations to fellow students.  After receiving an invitation, make sure to accept the invitation.

If the team you form has 6 to 10 members, no changes will be made through subsequent allocation.  Outside that range, adjustments will made to the team: teams of 5 people or less will have additional people allocated and teams of 11 members or more are split into two.  If you are not invited to a team or you do not accept an invitation to join a team, but wish to participate in the major group project, you must submit a form to request to be allocated to a team.  As in the small group project, there are two rounds of allocation.

\section{Assignment}
\tfaq[1]{0mm}{What choice of project assignment will we have in the major group project?}{Major~group~project}
\index{major~group~project!assignment}
\index{assignment!major~group~project}
Unlike the small group project, there will be multiple major group project assignments for teams to choose from.  There are three types of assignment, though they come with different eligibility requirements, as follows:
\begin{itemize}
\item\emph{Self-proposed project}.   Teams (not individuals!) can propose their own project.  To do so, you must have a complete team and submit a project proposal by the relevant deadline.  The entire team must agree to undertake this project.  The proposal must be approved. \\
\item\emph{Client-proposed project}.  These projects are proposed by a client who requires a software system to be developed to specific requirements.  If you undertake one of these projects, you will have several (at least three) meetings with the client.  Each client project will have limited capacity, depending on the client's availability to engage.  To apply for a client project, you must form a complete team, agree (collectively as a team) to bid for one or more client projects, and submit your agreed client-project preferences by the relevant deadline.  You can only undertake a client project if you are assigned one following the bidding process.\\
 \item\emph{Academic-proposed project}.  These are projects proposed by academics in the department.  The objectives tend to be relatively open-ended, allowing the team to design their own project.  There are no constraints on this type of project.  If your team is not eligible to undertake a self-proposed project or a client-proposed project, you must choose (collectively as a team) any one of the academic proposed projects.
 \end{itemize}

\subsection{Self-proposed project}
\tfaq[1]{0mm}{I have an idea for a self-proposed project.  Can I undertake this project instead of one proposed by staff or clients?}{Major~group~project}
If you have a good idea for a project that you, and the team you have formed, can complete within 10 weeks, then you can do a \emph{self-proposed project}\index{major~group~project!self-proposed}\index{self-proposed~project}.  The deadlines to do this are tight, so you need to start early.  You will need to assemble a complete team, write a proposal together, and submit it by the relevant deadline.   Every member of the team must have read the proposal and agreed to the project.  There are some further considerations: 
\begin{itemize}
\item The proposal must be for a new, self-contained software development project.  If any code or part of the system that you will be building already exist, you must declare what that is and share the code.  It is critical that the team is assessed on what it produces during the major group project period.
\item You must agree clear ``intellectual property'' arrangements\index{intellectual~property} for the software that will be developed in this project.  Do not assume that that is something that can be resolved later, as this can be very complicated.  I strongly recommend that the team agrees to release the software under a free\footnote{``Free'' as in ``free speech'', not as in ``free beer''.}.  This removes most, if not all, complications.  In light of this, it is a good idea to work on something you would enjoy doing, but not necessarily something you would like to commercialise.
\item I will review your project and return some feedback shortly after the deadline.  Where I have concerns, I will try to encourage you to proceed and suggest solutions if I can.  However, if your project involves work that is illegal, unethical, or in violation of College regulation, it will be rejected.  Projects that require ethics approval (e.g. a project that involves collecting new, personal data) are discouraged because the ethics approval process will delay the project too long.
\end{itemize}

Unless your self-proposed project has a particular client in mind, I recommend that the outcome of the project is released under a free software license.  Other arrangements, such as sharing intellectual property rights amongst the team or assigning it to one member of the team, risk causing conflict within the team during or after the project.  If your team does not wish to release the project under a free software license, the team should agree intellectual property arrangements for the project prior to starting the work.  I recommend you seek the advice of an experienced solicitor, as I am not qualified to provide legal advice.

\subsection{Client-proposed project}
\tfaq[1]{0mm}{What are client-proposed projects?}{Major~group~project}
We will offer one or more \emph{client-proposed projects}\index{major~group~project!client-proposed}\index{client-proposed~project}.  These are project proposed by a (normally external) client with specific objectives, requirements, constraints, and context.  The clients' ideas will have been scrutinised before offering these projects.  However, their ideas will not necessarily have been fully worked out.  
\begin{itemize}
\item  If you take on a client project, your team will have to work with the client to develop a suitable solution.  Client ideas are usually not fully worked out.  Some clients may have a limited appreciation of what is technically possible or how difficult different aspects of the work are.  If so, your team will have to advise them and negotiate how to proceed with the project.  Consequently, the requirements engineering for client-proposed projects tends to be substantially more complex compared to other types of project. 
\item The ``intellectual property'' arrangements\index{intellectual~property} for client-proposed projects vary.  Typically, the software is either owned fully by the client or it is released under a free software license.  
\item Capacity on client-proposed projects is limited.  If you are interested in one or more client-proposed projects, you must form a complete team, agree what projects you might like to do, and submit that team's preferences by the relevant deadline.  Projects will be allocated to team's based on the submitted preferences.  We cannot guarantee that all teams will be able to get a client-proposed project of their choice.
\end{itemize}

\subsection{Self-proposed client project}
\tfaq[1]{0mm}{Can I self-propose a project with an external client?}{Major~group~project}
If you know a potential client and wish to consider working with them, I would ask you to discuss this with me as soon as possible.  If they have a viable project, this project can be employed as a major group project, either as a self-proposed project or a client proposed project.  In either case, your team will have first right of refusal to undertake this particular project.  If the project is handled as a self-proposed project, your team would be the only one undertaking this project.  If it is handled as a client-proposed project, multiple teams would be allowed each to produce a solution independently.  As with other self-proposed or client-proposed projects, you must form a team prior to the relevant deadlines and ensure that the entire team is on board with the project.  In addition, I wish to contact the prospective client, so that I can discuss expectations with them.

\subsection{Academic-proposed project}
\tfaq[1]{0mm}{What are academic-proposed projects?}{Major~group~project}
\emph{Academic-proposed projects}\index{major~group~project!client-proposed}\index{academic-proposed~project} are the default type of project.  These are projects proposed by academics to produce an application or system to meet certain broad objectives.  Like the small group projects, the scope of these projects can be tailored somewhat to suit your team.  They will not come with ``client meetings'' or highly detailed requirements.  Instead, your team will need to design a product around the specified objectives that suits a particular end-user need.  This may require some independent research.

There are no capacity constraints on these projects.  If your team does not have a self-proposed or client-proposed project, your team must choose one of these.  All teams that have lecturer-allocated members in them must do an academic-proposed project of their choice. The outcome of the project is released under a free software license.


\section{Technology constraints}
\tfaq[1]{0mm}{What languages, frameworks, and technologies are allowed in the major group project?}{Major~group~project}
\index{major~group~project!technology~constraints}
\index{technology~constraints!major~group~project}
In the major group project, teams can choose the programming languages, frameworks, libraries, services, and tools they deem most suitable for the task at hand, provided the following constraints are met.
\begin{itemize}
\item They must be free (as in ``free beer'', not necessarily as in ``free speech'').  This means that teams must not pay for anything they use.
\item Team must use git with either GitHub.com's or King's College London's GitHub Enterprise service.  External clients may request deviations from this rule in advance of the project start and an exception is made to this rule \emph{only} for client-proposed projects where conventional approaches to using git are not possible.
\item Employing these tools must involve substantial coding and software design work.  For example, creating websites with a \ac{WYSIWYG} design tool is not permitted.
\end{itemize}

When choosing a technology stack, it is important to be aware of all the implications of your choice.  Consider the following questions.
\begin{itemize}
\item \emph{How will you meet the code cleanliness and design requirements for the project?} Certain frameworks and libraries promote sound design and code cleanliness, while others do not.  Make sure that the team is comfortable with the technology you plan to produce well-designed software and clean code.
\item \emph{How will you use this technology to produce an automated test suite?}  The marking criteria impose requirements on automated testing.  A good project will require you to produce a comprehensive test suite as well as a code coverage report.  Prior to committing to a particular technology, the team should identify the tools they will use to achieve this and ensure the team can learn to use them effectively.
\item \emph{How will you deploy your work?}  Some teams fail to consider this question before they are firmly committed to a technology and end up tied to a system that is very difficult to deploy.  It is essential to deploy your application.  Therefore, it is advisable to try deploying some software before committing to certain technologies.
\item \emph{How much effort is required to ensure the team becomes productive with a technology?}  Some teams choose a technology because one or two team member's strongly advocate for it.  However, this is a decision that affects everyone engaged with the work, especially the people on whom a technology choice imposes study requirements.  Do not pick a technology that team members will struggle to learn.
\item \emph{Is everyone able to install the tools needed for development?}  Some technologies come with requirements that some team members are unable to meet.  For example, it is obviously a bad idea to build an application with Apple's XCode if a member of the team does not have a Mac.
\end{itemize}

The requirements listed above tend to be easier to meet if teams build a web application.  Modern technology stacks to build web applications tend to come with good testing tools.  In deployment, teams have considerable control over the configuration of the (virtual) system the software runs on.  Web applications are inherently cross-platform.  The scope of a web-based information system is typically very scalable, enabling teams to adjust what they produce as circumstances change.  When building a web-application, make sure your system does not rely on paid-for services, such as payment handling services, and paid-for cloud computing infrastructure, such as Amazon Web Services.

If your team needs to or wishes to build a mobile application\index{mobile~application!native}, consider carefully whether this needs to be a native application.  Often, a web application with a front-end developed for mobile browsers can be nearly indistinguishable from a native application.  However, the web version will be easier to test and deploy.  You will not gain marks for choosing a technology stack that complicates your work.  Instead, as an engineer, you ought to choose the best tools for the job under the constraints you have been given.

Occasionally, a team chooses to build a computer game as their major group project.  While this can be a very interesting project to do, such a project comes with considerable risks and challenging.  It can be difficult to meet the testing requirements with certain modern game engines.  Moreover, a minimum viable product (MVP) for a game can be quite substantial, leaving only limited scope to scale down the application.  Often, 10 weeks is a very short time for an inexperienced team to produce a significant game.  Finally, a typical game requires multimedia resources and the production of such resources is not recognised in the marking scheme.  Thus, game development only a viable option for teams that are prepared to manage and face the risk involved.

Some teams seek to incorporate AI features into the systems that they build.  While such features can contribute to the project objectives, it is important to note that (i) this project is a software engineering project (not an AI project), and (ii) the members of the team will have only had limited exposure to AI.  Therefore, care must be taken that such features are sufficiently narrow in scope and ambition, so that they are feasible.  Moreover, certain sub-symbolic AI techniques require computational resources beyond what a typical consumer computer provides.  Do not expect high performance computing infrastructure to be available for your project, unless their need has been identified and resourced prior to the start of the project.  You may discuss this with the module organiser if you are unsure whether your ideas are achievable.  Do take care to ensure that the AI code produced meets the software engineering standards set out in the marking criteria. 


\section{Project management approach}
\tfaq[1]{0mm}{How should the major group project be project managed?}{Major~group~project}
\index{small~group~project!project~management~approach}
\index{project~management~approach!small~group~project}
Teams can use whichever project management approaches suit their team and project best.  There is no requirement to follow the lean and agile approaches required in the small group project.  Teams do not have to maintain a Kanban board.  However, the project management approaches of the small group project have been designed to avoid or mitigate many of the risks involved in student group projects.  If your team decides to deviate from this approach, consider the implications carefully.

To ensure it is possible to maintain at least basic levels of \emph{accountability}\index{accountability}, teams must meet following expectations (irrespective of the project management approach used): 
\begin{itemize}
\item Teams must have \emph{at least} one whole team meeting per week.  Everyone must attend every whole team meeting.  Team members must work on the tasks agreed with the team at team meetings, and only such tasks.  Team members not attending meeting, expecting colleagues to tell them what to do or choosing their own work, will not normally be considered full members of the team.  Team members must not decide what they work on without consultation of and agreement with the team.  Without this, coordination of work between team members becomes impossible.
\item Teams must use git version control with either GitHub.com's or King's College London's GitHub Enterprise service to version control their code base, their report document, and any prototype coding activities (including machine learning work).  Use separate remote repositories on GitHub for each item your project requires\footnote{Team Feedback encourages creating as few repositories as necessary.  While you should not create more repositories than necessary, each distinct component of your project can have its own repository.}.  Register all shared repositories on Team Feedback. Without this, it is impossible to obtain a view of individual engagement with the project.
\item Every member of the team must make demonstrable contributions to the project in every week of the project duration.  These contributions should normally be visible in the team's repository.  Without weekly delivery of contributions, it is difficult for a team to predict what a member can and will contribute.  This makes planning unnecessarily difficult.  The major marking scheme is based on this expectation: if a peer assessment raises significant concerns, and there is no evidence of weekly contributions, then the scheme is applied.
\item Everyone must contribute significantly and meaningfully \emph{both} to the report and the software.  Work on the report and the software targets different learning outcomes, and it is important that all team members attain both.  Moreover, the report must reflect the reflections of participants of software development project.  To achieve this, the people writing the report must be the same people that completed the project.  \marginnote{\vspace{-30mm}\newthought{Expectation}:\index{expectations!report~v~code~contributions} Everyone must contribute substantially to both code and report.  Do \emph{not} create specialist roles where certain individuals contribute to only one and not the other.  Contributions should be observable in the report and code repositories.}
\end{itemize}

\section{Deliverables}
\tfaq[1]{0mm}{What are the deliverables for the major group project?}{Major~group~project}
\index{major~group~project!deliverables}
\index{deliverables!major~group~project}
\marginnote{\newthought{Further reading}: On KEATS, navigate to \emph{Major group project $\rightarrow$ Handbook} to find a detailed description of the project deliverables.}
The team's deliverables at the end of the project consists of four parts:
\begin{enumerate}
\item \emph{Software and report}: On KEATS, you must submit a single ZIP file containing the following items:
  \begin{itemize}
  \item \emph{The software}: This includes everything needed to install, configure, run, test, and evaluate the application or system.  If the software is a system consisting of multiple components (e.g. a web application and a mobile application that interact with one another), the zip package must include a directory for each component (with everything needed to install, configure, run, test, and evaluate the component of the system).  For components containing a database, a database seeder/unseeder script should be provided.
  \item \emph{The project report}: This is a moderately substantial report that reflects on the project management and software engineering decisions that were made in the development of the software, explain their rationale, and relate it to the theory you were taught in the course.  The project report should be jointly authored by all team members.  To produce this document, teams will be given a \LaTeX\index{latex@\LaTeX} document with detailed instructions on what the report must contain.   
  \item A single \emph{developer's manual}:  Ensure that this is a file named \texttt{developers-manual.pdf} and that it resides at the root of the application.  This file must contain detailed instructions how to install, configure, run, and test the software, and each of its components.  For components containing a database, explain how to generate data for the database.  It is good practice to employ \emph{virtual environments}\index{virtual~environment} and build automation as much as possible.  Beware that your submission may rely on specific versions of libraries and frameworks, so make sure your specifications are sufficiently detailed. 
  \item A \texttt{README.md} file: This file must include the title of the project and the name of the software, the names of all authors of the software, a list of all significant parts of the source code written by others that you employed directly or relied on heavily when writing this software and the locations of this source material.  Think of this as the "reference list" for your source code, and the location where the software or software component is deployed and sufficient information to access it.  The latter includes access credentials for the different types of user who may employ the software.
  \end{itemize}  
\item \emph{Screencast}: On KEATS, you must submit a single video file containing a screencast demonstrating the software.  This is a video showing all (significant) features of the software.  It is important that the demonstration includes each software component and the typical interactions of each type of operator (user) with the system.  Minor features, such as I/O validation, do not need to be included throughout the video.  The screencast should be submitted as a single video file (even if the system contains multiple software components). Your screencast should come with clear explanations of what you are demonstrating.  The screencast should be narrated, as that tends to be the most efficient way to incorporate the explanations.  The video should be saved MP4/MPEG-4 format.  If you are unable to do this, common alternative video formats are acceptable.  However, do not use AVI as that tends to create a very large file!  You can submit a ZIP file provided it contains only a single video file.  Bear in mind that video tend to be highly compressed already: zip algorithms do not tend to reduce the file size significantly.
\item A deployed system: Teams must make a deployed version of the application available\footnote{Unless you undertake a project with an external client who, for valid commercial reasons, does not allow this.  This is normally known in advance of project selection.}  Ideally, the deployed system is available from within a web browser on  a URL.   Most web applications and mobile applications can be deployed that way.  Generally speaking, a deployed system is one that is readily available for execution without the need to set up a comprehensive development environment.  This is quite a vague statement, but it is impossible to be more precise as your options for deployment will vary considerably from project to project.  
\item Team management data on Team Feedback: Team Feedback must have access to your team's Git repositories.  All team meetings must have been minuted within 24 hours of each meeting taking place.
\end{enumerate}

\section{Assessment}
\subsection{Team marking scheme}
\index{major~group~project!team~marking~scheme}
\index{team~marking~scheme!major~group~project}
The team marking scheme for the major group project employs the same approach as that of the small group project.  The mark scheme is divided into 10 percentage point intervals, from 10\% to 100\%.  To achieve a particular mark, say 60\%, the team's work must meet all the requirements associated with that mark \emph{and all the requirements that precede it}.  If a submission does not meet a particular criterion (say at the 60\% point), the team will not attain the mark associated with that criterion (i.e. their mark will be less than 60\%). 

Markers will not consider achievements at a given mark point unless all expectations of all criteria at the lower mark point have been achieved.  If a team achieves some of the criteria at a particular mark point, they will receive some marks for this, proportional to the number of criteria attained at the higher level).  However, the mark will be capped below the grade point for which the submission does not attain all expectations.  Therefore, it is crucial that your team understands all marking criteria, and gets the basics right before trying to attain higher level expectations!  

\paragraph{Band I: 0--20\%}
\tfaq[1]{0mm}{What criteria must be met in order to attain a team mark in the 0--20\% range (band I) in the major group project?}{Major~group~project}
\begin{itemize}[align=left, labelwidth=2.5em, labelsep=1em, leftmargin=3.5em]
\item[I.1.1]\criterion{Delivery}{\necessary}{1}{The submission is structured as required.  All files and directories, including the top-level directory, meet the required naming conventions.}{To meet this requirement, make sure you read the specification of the deliverables carefully.  Prevent errors by maintaining the project structure using the required specifications from the very start.  Leave sufficient time to double or triple check the submission before uploading.}
\item[I.1.2]\criterion{Delivery}{\necessary}{2}{All required content is included in the submission as specified in the checklist of deliverables\footnote{The checklist of deliverables will be provided via KEATS under the Major Group Project topic.}.}{Keep track of the list of deliverable and what you have completed in a shared document.}
\item[I.2.1]\criterion{Design}{\critical}{1}{The source code includes functional data structures or domain models that represent core entities and support application logic.}{}
\item[I.2.2]\criterion{Design}{\critical}{1}{The source code includes functional components or handlers that respond to user interactions or requests.}{}
\item[I.2.3]\criterion{Design}{\critical}{1}{The source code includes presentation-layer templates or components that render dynamic content based on application data.}{}
\item[I.2.4]\criterion{Design}{\necessary}{1}{The submission includes a diagram modelling the structure of the system that has been produced.  A UML structural diagram (a class diagram, a component diagram, or a deployment diagram) is expected.  Where the team is unable or unwilling to employ UML, the diagram should be accompanied by a legend describing the components.}{}
\item[I.4.1]\criterion{Functionality}{\critical}{1}{The application delivers some form of content to the end user.  This content may be static content or views includes static content or views that are accessible without any dynamic data processing.}{}
\item[I.4.2]\criterion{Functionality}{\critical}{2}{The application includes dynamic interfaces or views that display data retrieved from a persistent storage layer (e.g. a database or a file) or retrieved from an online data source.}{}
\item[I.4.3]\criterion{Functionality}{\critical}{3}{The application includes a number of usable features corresponding to at least the equivalent to two one-person-week user stories per effective team member.}{Use the full development period.  This will ensure that this requirement is met.}
\item[I.6.1]\criterion{Version Control}{\critical}{1}{The project's git repository is accessible via Team Feedback.}{Make the shared repository accessible as soon as possible.}
\item[I.6.2]\criterion{Version Control}{\critical}{1}{The repository shows a history of mostly small commits throughout development.}{Expect team members to make regular commits when working on tasks and push these immediately to the shared repository.  Use this information to assess when team members are actively working on the project.  If you enforce this expectation, the repository will naturally meet this requirement.}
\item[I.6.3]\criterion{Version Control}{\necessary}{2}{Commits include informative messages that adhere to the requirements for commit messages set out in \ref{sect:communication:version-control} of the Module Handbook.}{Review commit messages regularly, or when merging work at the very latest, }
\item[I.8.1]\criterion{}{\necessary}{1}{Team Feedback contains a record of meetings (at least one per week while the team is working on the project).  Each meeting is recorded on Team Feedback within less than one week after the meeting took place.  Each meeting record includes an accurate attendance list.}{Assign a team member the responsibility to take meeting minutes.  Take attendance and record a first draft of the minutes during the minutes rather than postpone this until a later date.}
\item[I.8.2]\criterion{}{\necessary}{2}{Each meeting record meets the expectations set out in Section~\ref{sect:communication:team-meetings}, including an agenda, a clear outline of key decisions made at the meeting, and log of actions arising from the meeting (other than software development task allocations assigned via a project management tool, such as Trello).}{}
\end{itemize}

\paragraph{Band II: 20--40\%}
\tfaq[1]{0mm}{What criteria must be met in order to attain a team mark in the 20--40\% range (band II) in the major group project?}{Major~group~project}

The criteria of this band are assessed \emph{only if all \critical} requirements of band I are met in full.  
\begin{itemize}[align=left, labelwidth=2.5em, labelsep=1em, leftmargin=3.5em]
\item[II.1.1]\criterion{Delivery}{\necessary}{1}{The submission includes automated setup scripts or configuration files to initialise the development environment and install dependencies.  Unless asked otherwise, you will be using Nix.}{Further information on using Nix or related tools will be released in January 2026.}
\item[II.1.2]\criterion{Delivery}{\necessary}{1}{There are predefined user access credentials specified in the \texttt{README.md} file that the examiners can access the application.}{Do not implement two-factor authentication or CAPTCHA unless required to do so by a client in a client-proposed project.  Such features make accessing your application more time-consuming for markers and should not be necessary for a demonstration application.}
\item[II.1.3]\criterion{Delivery}{\necessary}{1}{The submission supports database setup and seeding/unseeding via automation commands.}{}
\item[II.1.4]\criterion{Delivery}{\necessary}{1}{The submission supports running tests and generating a coverage report via automation commands. The submission includes the original coverage report(s) as produced by the team's chosen code coverage tools.}{The availability and easy of use of automated testing and code coverage tools must be a consideration in the decision as to what technology stack to use in the project.  Do not commit a technology stack unless this is resolved.}
\item[II.1.5]\criterion{Delivery}{\necessary}{1}{The submission supports running the application via automation commands.}{}
\item[II.2]\criterion{Design}{\critical}{3}{The design diagrams are free from major errors and reflect the application developed.}{}
\item[II.3]\criterion{Code (Source)}{\critical}{3}{The code is free from significant defects or major bugs.}{Test your code regularly and before each merge.}
\item[II.4.1]\criterion{Functionality}{\critical}{1}{The application allows users to input and submit new data through appropriate user interfaces or interaction mechanisms.}{}
\item[II.4.2]\criterion{Functionality}{\critical}{1}{The application provides functionality to retrieve and display existing data to users.}{}
\item[II.4.3]\criterion{Functionality}{\critical}{1}{The application allows users to modify existing records or content through appropriate interfaces.}{}
\item[II.4.4]\criterion{Functionality}{\critical}{1}{The application includes functionality for users to remove or deactivate data as needed.}{}
\item[II.4.5]\criterion{Functionality}{\critical}{1}{The application integrates data management features into cohesive workflows that support user goals.}{}
\item[II.4.6]\criterion{Functionality}{\critical}{1}{The scope and scale of the application is adequate for the team's effective size.}{}
\item[II.5]\criterion{Testing}{\critical}{3}{There are at least two automated tests per effective team member.}{This criterion should be easy to meet if the team produce tests alongside source code.}
\end{itemize}

\paragraph{Band III: 40--60\%}
\tfaq[1]{0mm}{What criteria must be met in order to attain a team mark in the 40--60\% range (band III) in the major group project?}{Major~group~project}

The criteria of this band are assessed \emph{only if all \critical} requirements of bands I and II are met in full.  
\begin{itemize}[align=left, labelwidth=2.5em, labelsep=1em, leftmargin=3.5em]
\item[III.1.1]\criterion{Delivery}{\necessary}{1}{The submission includes clear reused software references or AI generated code in README.md.}{}
\item[III.1.2]\criterion{Delivery}{\necessary}{1}{The application can be installed and run without manual fixes beyond documented steps.}{}
\item[III.2]\criterion{Design}{\critical}{3}{The software design separates concerns through the use of appropriate modules, such as class-based components or well-structured functions.  Where a framework is used, the framework structural conventions are broadly adhered to.  Where no framework is used, the team introduced a basic, sensible organisation to software.}{}
\item[III.3.1]\criterion{Code (Source)}{\critical}{2}{The code mostly follows consistent naming conventions.}{}
\item[III.3.2]\criterion{Code (Source)}{\critical}{2}{The code contains no egregious code smells that an average team should be able to pick up via superficial reviews (e.g., very deeply nested code, very long functions, egregious examples of code repetition).}{}
\item[III.4.1]\criterion{Functionality}{\critical}{2}{The application supports multiple related features integrated into a cohesive workflow.}{}
\item[III.4.2]\criterion{Functionality}{\critical}{2}{Features are fully functional and tested manually (no broken paths).}{}
\item[III.4.3]\criterion{Functionality}{\critical}{2}{The application handles basic error cases gracefully (e.g., invalid input).}{}
\item[III.5.1]\criterion{Testing}{\critical}{2}{The project includes unit tests for critical features.}{}
\item[III.5.2]\criterion{Testing}{\critical}{2}{The test suite achieves coverage greater than 50\%.}{Use a code coverage tool.}
\item[III.8]\criterion{Project management}{\necessary}{1}{The team maintains updated task tracking (e.g., backlog, progress board). This is made available either through a Trello board connected to Team Feedback, or other evidence that needs to be submitted in the \texttt{appendixes.zip} file, so that examiners have access.}{}
\end{itemize}

\paragraph{Band IV: 60--80\%}
\tfaq[1]{0mm}{What criteria must be met in order to attain a team mark in the 60--80\% range (band IV) in the major group project?}{Major~group~project}

The criteria of this band are assessed \emph{only if all \critical} requirements of bands I, II, and III are met in full.  
\begin{itemize}[align=left, labelwidth=2.5em, labelsep=1em, leftmargin=3.5em]
\item[IV.1]\criterion{Delivery}{\critical}{0}{All \necessary criteria from bands I, II, and III were met (precondition for Band V eligibility).}{}
\item[IV.2.1]\criterion{Design}{\critical}{1}{Code adheres to separation of concerns and avoids bloated or multi-purpose handlers.}{}
\item[IV.2.2]\criterion{Design}{\critical}{2}{Shared logic or rules (e.g., access control, validation) are abstracted and reused across components to avoid duplication.}{}
\item[IV.3.1]\criterion{Code (Source)}{\critical}{1}{Variables, functions, and classes have clear, descriptive names.}{}
\item[IV.3.2]\criterion{Code (Source)}{\critical}{1}{Consistent code layout throughout the project.}{}
\item[IV.3.3]\criterion{Code (Source)}{\critical}{1}{No function/method exceeds 25 lines.  No function/method has more than 2 levels of nesting.}{}
\item[IV.3.5]\criterion{Code (Source)}{\critical}{1}{The code base is mostly DRY (significant repetition avoided).}{}
\item[IV.3.6]\criterion{Code (Templates)}{\critical}{1}{Templates are structured to minimise repetition through modular and reusable components. (e.g.: Common layout elements such as headers, footers, navigation are abstracted into separate files or components).}{}
\item[IV.3.7]\criterion{Code (Templates)}{\critical}{1}{Presentation is separated from structure and logic (e.g.: no inline styling, CSS classes used instead; design systems are used).}{}
\item[IV.4.1]\criterion{Functionality}{\critical}{2}{The application supports an ambitious range of objectives.}{}
\item[IV.4.2]\criterion{Functionality}{\critical}{1}{The feature set is cohesive, avoiding isolated features that don't contribute to objectives.}{}
\item[IV.4.3]\criterion{Functionality}{\critical}{1}{Features are fully developed and offer an intuitive, polished interface.  Expedient implementation at the expense of usability is generally avoided.  For example, dates/times follow UK conventions and are intuitive.  Records can be identified without exposing internal IDs (e.g., primary keys).  Lists include pagination, ordering, and searching.}{}
\item[IV.4.4]\criterion{Functionality}{\critical}{1}{Features provide flexibility for end users (e.g., multiple ways to interact, customisation, admin parameters, preferences etc.).}{}
\item[IV.5.1]\criterion{Testing}{\critical}{1}{The project includes a comprehensive test suite.}{}
\item[IV.5.2]\criterion{Testing}{\critical}{2}{The test suite achieves impeccable statement and branch coverage.}{}
\item[IV.5.4]\criterion{Testing}{\critical}{1}{All tests pass and there is evidence of manual tests.}{}
\item[IV.7]\criterion{Management}{\critical}{2}{The UI is consistent across screens (look and feel).  Terminology and language are consistent throughout the interface.  Equivalent controls appear in the same place with the same look and feel.}{}
\end{itemize}

\paragraph{Band V: 80--100\%}
\tfaq[1]{0mm}{What criteria must be met in order to attain a team mark in the 80--100\% range (band V) in the major group project?}{Major~group~project}

The criteria of this band are assessed \emph{only if all} requirements (\critical \emph{and} \necessary) of bands I, II, III, and IV are met in full.  
\begin{itemize}[align=left, labelwidth=2.5em, labelsep=1em, leftmargin=3.5em]
\item[V.2.1]\criterion{Design}{\critical}{1}{The design achieves high cohesion across components.}{}
\item[V.2.2]\criterion{Design}{\critical}{1}{The design achieves low coupling across components.}{}
\item[V.2.3]\criterion{Design}{\critical}{1}{Classes have limited responsibility (ideally single responsibility).  Functions and methods do one thing only.}{}
\item[V.3.1]\criterion{Code (Source)}{\critical}{1}{Naming is consistent and meaningful throughout the application.}{}
\item[V.3.2]\criterion{Code (Source)}{\critical}{1}{Functions and methods are extremely short (15 lines or less).  Functions and methods have no more than 1 level of nesting.}{}
\item[V.3.4]\criterion{Code (Source)}{\critical}{1}{The code includes no repetition (fully DRY).}{}
\item[V.3.5]\criterion{Code (Test)}{\critical}{1}{Test code uses clear, descriptive, and consistent names.}{}
\item[V.3.6]\criterion{Code (Test)}{\critical}{1}{Test code repetition is minimal.}{}
\item[V.3.7]\criterion{Code (Test)}{\critical}{1}{Test function/method bodies contain 25 lines or less with no more than 2 levels of nesting.}{}
\item[V.4.1]\criterion{Functionality}{\critical}{3}{The application is very ambitious in scope for the team's effective size.}{}
\item[V.4.2]\criterion{Functionality}{\critical}{3}{The application is exceptionally polished in terms of usability and completeness.}{}
\item[V.5.1]\criterion{Testing}{\critical}{2}{Test suites are carefully designed to cover a comprehensive range of input and output partitions.}{}
\item[V.5.2]\criterion{Testing}{\critical}{1}{Test suites address a wide range of potential error causes.}{}
\item[V.7]\criterion{Management}{\critical}{2}{The team's time management was excellent throughout the project.  The final three days of the project were free from significant development activity (based on commit stats).}{}
\end{itemize}


\subsection{Major mark correction}
\tfaq[1]{0mm}{Under what conditions would my individual major group project mark be a severe reduction on the team's mark?}{Major~group~project}
\index{major~group~project!major~mark~correction}
\index{major~mark~correction!major~group~project}
The major (individual) mark correction scheme of the major group is similar to that of the small group project, with some small differences.  As for the small group project, there is a correction scheme based on contribution statistics derived from GitHub and a correction scheme based on meeting attendance.  The major individual mark correction (if there is any), is the highest correction produced by both schemes.  
\begin{table}[ht]
\checkoddpage \ifoddpage \forcerectofloat \else \forceversofloat \fi
  \centering
  \fontfamily{ppl}\selectfont
  \begin{tabular}{p{30mm} p{30mm} p{30mm}}
    \toprule
    \#Weeks of minimum contribution & \#Weeks of used minimum contribution & Mark correction\\
    \midrule
    $n-2$ or more & 3 or more & no effect\\
    & 2 & -10\%\\
    & 1 & -20\%\\
    & 0 & -40\%\\
    \midrule
    $n-3$ & 4 or more & no effect\\
    & 3 & -10\%\\
    & 2 & -20\%\\
    & 1 & -30\%\\
    & 0 & -50\%\\
    \midrule
    $n-4$ & 4 or more & -10\%\\
    & 3 & -20\%\\
    & 2 & -30\%\\
    & 1 & -40\%\\
    & 0 & -60\%\\
    \midrule
    $n-5$ & 4 or more & -20\%\\
    & 3 & -30\%\\
    & 2 & -40\%\\
    & 1 & -50\%\\
    & 0 & -70\%\\
    \midrule
    $n-6$ & 4 & -30\%\\
    & 3 & -40\%\\
    & 2 & -50\%\\
    & 1 & -60\%\\
    & 0 & -80\%\\
    \midrule
    $n-7$ & 3 & -50\%\\
    & 2 & -60\%\\
    & 1 & -70\%\\
    & 0 & -90\%\\
    \midrule
    $n-8$ & 2 & -70\%\\
    & 1 & -80\%\\
    & 0 & -90\%\\
    \midrule
    $n-9$ & 1 & -80\%\\
    & 0 & -95\%\\
    \midrule
    0 & 0 & -100\%\\
    \bottomrule
  \end{tabular}
  \caption{Major individual mark correction for contribution (code and report) in teams with a development period of $n$ weeks.  Weeks are defined as the Monday-Sunday intervals between the start and end of the project.}
  \label{tab:major-group-project:major-correction:code-correction}
  %\zsavepos{pos:normaltab}
\end{table}

Due to the potential variety of projects in the major group project, it can be difficult to assess whether team members contributed every week to the project.  Team members should ensure their work is shared and tracked via a remote repository hosted by GitHub.  As this is not always consistently done, \emph{no major mark correction based on contribution concerns is applied unless there is at least one peer assessment that raises significant concerns}.  If one or more peer assessments about a team member are very poor, the contributions of that team member recorded on the registered team repositories are reviewed.  If they fall short of the expectation of weekly contributions throughout the team's project period, a correction is applied.  Table~\ref{tab:major-group-project:major-correction:code-correction} summarises the mark scheme based on contributions.


\begin{table}[ht]
\checkoddpage \ifoddpage \forcerectofloat \else \forceversofloat \fi
  \centering
  \fontfamily{ppl}\selectfont
  \begin{tabular}{p{60mm} p{30mm}}
    \toprule
    \% of meetings where team member is ``present'' or ``5 minutes late'' & Mark correction\\
    \midrule
    80\% or above & no effect\\
    60\% -- 79\% & -20\%\\
    40\% -- 59\% & -50\%\\
    20\% -- 39\% & -80\%\\
    less than 20\% & -100\%\\
    \bottomrule
  \end{tabular}
  \caption{Major individual mark correction for meeting attendance. }
  \label{tab:major-group-project:major-correction:attendance}
  %\zsavepos{pos:normaltab}
\end{table}

The major individual mark correction scheme for attendance is summarised in Table~\ref{tab:major-group-project:major-correction:attendance}.  It is identical to that of the small group project.

\subsection{Minor mark redistribution}
\index{major~group~project!minor~mark~redistribution}
\index{minor~mark~redistribution!major~group~project}
\tfaq[1]{0mm}{If my individual mark is not severely reduced, do I receive the same mark as my team members in the major group project?}{Major~group~project}
The minor mark redistribution scheme for the major group project operates in exactly the same manner as that of the small group project.  This scheme is explained on page~\ref{sect:sgp:minor-mark-redistribution}.

\subsection{Peer assessments}
\tfaq[1]{0mm}{How are the peer assessments marked in the small group project?}{Small~group~project}
\index{major~group~project!peer~assessment~mark}
\index{peer~assessments!major~group~project}

The peer assessments you write about your team mates are marked directly, focussing on the open-ended feedback text that you write.  This contributes towards 5\% of the small group project mark.  The peer assessment marking scheme in the major group project is similar to that of the small group project, except that the scheme is slightly stricter.  The following assessment criteria are used to produce this mark:
\begin{itemize}[align=left, labelwidth=2.5em, labelsep=1em, leftmargin=3.5em]
\item[0\%]  The feedback text contains no meaningful write-up.
\item[20\%]  A peer assessment was submitted for almost every team member.  The feedback text of each submitted peer assessment contains at least a single sentence explaining the scores for that peer assessment.
\item[40\%]  A peer assessment was submitted for every team member.  Each peer assessment includes at least a 3--4 sentence explanation for the peer assessment scores.  
\item[60\%]  A peer assessment was submitted for every team member.  Each peer assessment includes at least a 3--4 sentence explanation for the peer assessment scores, recognising strengths and weaknesses, as well as some valid constructive feedback that, when acted on, would help the reviewee in future projects.   The text is largely free from grammatical errors.
\item[80\%]  A peer assessment was submitted for every team member.  Each peer assessment includes a thorough explanation for the peer assessment scores,  recognising strengths and weaknesses, as well as valid constructive feedback that, when acted on, would help the reviewee in future projects.  The peer assessment is written thoughtfully, and internally cohesive.  The constructive feedback logically follows from the explanation.  Some ideas are rooted in project management theory.  Some are insightful.  The text is free from grammatical errors.
\item[100\%]  An unusually extensive peer assessment was submitted for every team member.  Each peer assessment includes a thorough explanation for the peer assessment scores,  recognising strengths and weaknesses, as well as valid constructive feedback that, when acted on, would help the reviewee in future projects.  The peer assessment is written thoughtfully, and internally cohesive.  The constructive feedback logically follows from the explanation.  The ideas are firmly rooted in project management theory and insightful throughout.  The text is free from grammatical errors. 
\end{itemize}

In some teams, you may have a team member who failed to engage with the team at all: i.e. they did not attend any meetings or produce any work.  Such a person may remain in your team if they signed up to participate in the small group project and did not defer early on.  Obviously, there is little to write about in such a situation.  The feedback to write for such a peer can be limited to a single sentence explaining the lack of engagement without affecting your peer assessment mark.
If a person's engagement was very limited, but they attended some meetings or produced some work, focus on your interactions with them.